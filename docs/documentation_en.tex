% Documentation following FMI template (report, Times, 1.5 line spacing)
% Authors: Ingrid Corobana, Teodora Nae
% Recommended compilation: xelatex documentation_en.tex

\documentclass[12pt,a4paper]{report}

% Font and language support (FMI template)
\usepackage{fontspec}
\usepackage{times}
\usepackage{polyglossia}
\setdefaultlanguage{english}
\setotherlanguages{romanian}

\usepackage{geometry}
\usepackage{setspace}
\doublespacing
\geometry{margin=2.5cm}
\usepackage{fancyhdr}
\usepackage{graphicx}
\usepackage{amsmath}
\usepackage{amssymb}
\usepackage{algorithm}
\usepackage{algpseudocode}
\usepackage{listings}
\usepackage{xcolor}
\usepackage{hyperref}
\usepackage{tikz}
\usepackage{array}
\usepackage{booktabs}
\usepackage{float}
\usepackage{caption}
\usepackage{pdfpages} % to include pipeline_diagrams.pdf

\geometry{
    a4paper,
    left=2.5cm,
    right=2.5cm,
    top=2.5cm,
    bottom=2.5cm,
    headheight=14pt
}

\pagestyle{fancy}
\setlength{\headheight}{14.49998pt}
\fancyhf{}
\fancyhead[L]{\textit{CFAR-STFT Detection}}
\fancyhead[R]{\thepage}
\renewcommand{\headrulewidth}{0.4pt}

\definecolor{stftblue}{RGB}{66, 133, 244}
\definecolor{cfarorange}{RGB}{255, 152, 0}
\definecolor{dbscangreen}{RGB}{76, 175, 80}
\definecolor{darkblue}{RGB}{25, 50, 100}
\definecolor{codegray}{RGB}{240, 240, 240}

\hypersetup{
    colorlinks=true,
    linkcolor=darkblue,
    urlcolor=stftblue,
    bookmarksnumbered=true,
    pdftitle={CFAR-STFT Documentation}
}

\lstset{
    language=Python,
    basicstyle=\ttfamily\small,
    backgroundcolor=\color{codegray},
    breaklines=true,
    keywordstyle=\color{blue},
    commentstyle=\color{gray},
    stringstyle=\color{red},
    identifierstyle=\color{darkblue},
    numbers=left,
    numberstyle=\tiny\color{gray}
}

\renewcommand{\algorithmicrequire}{\textbf{Input:}}
\renewcommand{\algorithmicensure}{\textbf{Output:}}

\title{
    \textbf{\Large Analysis of Radar Signals in the Presence of Sea Clutter}\\[0.5em]
    \normalsize CFAR-STFT Approach with Experiments on Synthetic and Real Data
}

\author{
    Ingrid Corobana \quad Teodora Nae\\
    \small \textit{Signal Processing Course}\\[0.3em]
    \href{https://github.com/dirgnic/Radar_Detection_STFT}{\texttt{github.com/dirgnic/Radar\_Detection\_STFT}}\footnote{Repository: \url{https://github.com/dirgnic/Radar\_Detection\_STFT}}
}

\date{\today}

\begin{document}

\maketitle

\begin{abstract}
This document presents a complete implementation of the CFAR-STFT algorithm proposed by Abratkiewicz (2022) for detection and recovery of radar signals in the presence of noise and clutter. The algorithm combines Short-Time Fourier Transform (STFT), adaptive 2D CFAR detection, DBSCAN clustering, and geodesic dilation to reconstruct radar signals with high fidelity.

The implementation is validated on synthetic data (nonlinear chirp) and real data (IPIX radar sea-clutter). On the controlled synthetic signal, the algorithm detects the component of interest in all 100 Monte Carlo runs (100\% detection rate). The Reconstruction Quality Factor (RQF) varies from 7.28 dB at SNR=5dB to 29.17 dB at SNR=30dB.

\textbf{Key contribution}: We adapted the algorithm for real sea clutter by implementing \textbf{K-distribution thresholds} (instead of Gaussian), \textbf{fractal/Hurst exponent boost} for weak target detection, and \textbf{asymmetric DBSCAN} for vertical line clustering.

Source code available at: \url{https://github.com/dirgnic/Radar_Detection_STFT}
\end{abstract}

\newpage
\tableofcontents
\newpage

% Pipeline diagrams section
\section*{Pipeline Diagrams CFAR-STFT}
\addcontentsline{toc}{section}{Pipeline Diagrams CFAR-STFT}

\includepdf[pages=1,scale=0.95,offset=0 -20]{pipeline_diagrams.pdf}
\newpage
\includepdf[pages=2-,scale=0.95,offset=0 -20]{pipeline_diagrams.pdf}

\newpage

\section{Introduction}

Reliable detection of radar signals in the presence of noise and clutter remains a central problem in modern radar systems. Traditional adaptive detection methods (CFAR, constant false-alarm rate) are limited to the frequency domain, losing important temporal information.

Abratkiewicz (2022)\footnote{Abratkiewicz, K. (2022). Radar Detection-Inspired Signal Retrieval from the Short-Time Fourier Transform. Sensors, 22(16), 5954.} proposes an innovative approach that exploits the time-frequency structure of radar signals to improve both detection and recovery of signal components.

\subsection{Project Objectives}

This project pursues the following objectives:

\begin{enumerate}
    \item Complete implementation of the CFAR-STFT algorithm in Python
    \item Validation on synthetic data (nonlinear chirp according to Eq. 14 from the paper)
    \item Testing on real data (IPIX radar with complex sea-clutter)
    \item Doppler analysis for object velocity estimation
    \item Detailed documentation and complete reproducibility
    \item Results validation: detection and reconstruction
    \item \textbf{Adaptation for sea clutter}: K-distribution, fractal boost, asymmetric DBSCAN
\end{enumerate}

\subsection{Document Structure}

The document is organized as follows:

\begin{itemize}
    \item \textbf{Section 2}: Theoretical foundations and essential mathematical formulas
    \item \textbf{Section 3}: Detailed description of all algorithm steps with pseudocode
    \item \textbf{Section 4}: Complete experimental results and analysis
    \item \textbf{Section 5}: Sea clutter adaptations (K-distribution, Hurst, asymmetric DBSCAN)
    \item \textbf{Section 6}: Implementation details with code references
    \item \textbf{Section 7}: Conclusions and future perspectives
\end{itemize}

\section{Theoretical Foundations}

\subsection{Short-Time Fourier Transform (STFT)}

STFT is the foundation of the algorithm, providing a time-frequency representation of the signal:

\begin{equation}
X(k,n) = \sum_{m=0}^{N-1} x(m) \cdot w(m - nH) \cdot e^{-j2\pi km/N}
\end{equation}

where:
\begin{itemize}
    \item $x(m)$ --- input signal
    \item $w(m)$ --- window (Gaussian, $\sigma=8$ bins)
    \item $n$ --- time index (hop between windows)
    \item $k$ --- frequency index
    \item $N$ --- FFT length (512 samples)
    \item $H$ --- hop between windows (256 samples, 50\% overlap)
\end{itemize}

\textbf{Gaussian Window}: 
\begin{equation}
w(m) = e^{-m^2/(2\sigma^2)} \quad \text{with} \quad \sigma=8 \text{ bins}
\end{equation}

The Gaussian window is chosen for its spectral leakage minimization properties.

\subsection{Adaptive CFAR 2D Detection}

CFAR (Constant False-Alarm Rate) is a classic method that adapts the detection threshold locally based on the local noise level:

\begin{equation}
H(k,n) = \begin{cases}
1 & \text{if } |X(k,n)|^2 > \lambda \cdot \mathcal{N}(k,n) \\
0 & \text{otherwise}
\end{cases}
\end{equation}

where $\mathcal{N}(k,n)$ is an estimate of the local noise level. We use \textbf{GOCA-CFAR} (Greatest-Of Cell Averaging), which divides training cells into 4 quadrants and takes the maximum for robustness to non-homogeneous clutter.

\subsection{DBSCAN Clustering}

After CFAR, detected points are grouped using DBSCAN density-based clustering. We use an \textbf{asymmetric distance metric} with 3x tolerance in the frequency direction to properly cluster vertical target signatures (see Section~\ref{sec:adaptations}).

\section{Complete Algorithm Description}

\subsection{General Pipeline}

The complete algorithm consists of five main steps:

\begin{enumerate}
    \item Computing STFT with Gaussian window
    \item Adaptive 2D CFAR detection in the time-frequency plane
    \item DBSCAN clustering of detected points
    \item Geodesic dilation of the detection mask
    \item Inverse reconstruction (iSTFT) with mask
\end{enumerate}

\subsubsection{Step 1: STFT Computation}

\begin{algorithm}[H]
\caption{STFT Computation with Gaussian Window}
\begin{algorithmic}[1]
\Require Input signal $x[n]$, FFT length $N_{fft}=512$, hop $H=256$, $\sigma=8$
\Ensure STFT matrix $X_{stft} \in \mathbb{C}^{N_f \times N_t}$
\State $N_t \gets \lceil (len(x) - N_{fft}) / H \rceil + 1$
\State $X_{stft} \gets \text{zeros}(N_{fft}/2 + 1, N_t)$ \Comment{One-sided}
\State Precompute Gaussian window: $w[m] \gets e^{-m^2/(2\sigma^2)}$
\For{$n \gets 0$ to $N_t - 1$}
    \State Extract window: $x_n \gets x[nH : nH + N_{fft}]$
    \State Apply window: $x_w \gets x_n \odot w$
    \State Compute FFT: $X_n \gets \text{fft}(x_w, N_{fft})$
    \State Store one-sided: $X_{stft}[:, n] \gets X_n[0 : N_{fft}/2 + 1]$
\EndFor
\State Normalize: $X_{stft} \gets X_{stft} / \sum_m w[m]^2$
\State \Return $X_{stft}$
\end{algorithmic}
\end{algorithm}

\subsubsection{Step 2: CFAR 2D Detection}

\begin{algorithm}[H]
\caption{CFAR 2D Detection (GOCA)}
\begin{algorithmic}[1]
\Require STFT matrix $X_{stft}$, $P_f=0.001$, $N_G=3$, $N_T=12$
\Ensure Binary detection mask $H \in \{0,1\}^{N_f \times N_t}$
\State $H \gets \text{zeros}(N_f, N_t)$
\State $N_f \gets $ rows($X_{stft}$), $N_t \gets $ cols($X_{stft}$)
\For{$k \gets N_G + N_T$ to $N_f - N_G - N_T - 1$}
    \For{$n \gets N_G + N_T$ to $N_t - N_G - N_T - 1$}
        \State Extract Training Cells in 4 quadrants around $(k,n)$
        \State $\mu_i \gets \text{mean}(\text{quadrant}_i)$ for $i \in \{1,2,3,4\}$
        \State $\mathcal{N}_{local} \gets \max(\mu_1, \mu_2, \mu_3, \mu_4)$ \Comment{GOCA}
        \State $\lambda \gets R(P_f) \cdot \mathcal{N}_{local}$
        \If{$|X_{stft}(k,n)|^2 > \lambda$}
            \State $H(k,n) \gets 1$
        \EndIf
    \EndFor
\EndFor
\State \Return $H$
\end{algorithmic}
\end{algorithm}

\subsubsection{Step 3: DBSCAN Clustering}

\begin{algorithm}[H]
\caption{DBSCAN Clustering in Time-Frequency Plane}
\begin{algorithmic}[1]
\Require Detected points $\{(f_i, t_i)\}_{i=1}^{N_p}$, $\varepsilon=8$, minSamples$=5$
\Ensure Cluster labels $\text{labels} \in \mathbb{Z}$
\State $\text{labels} \gets -1 \cdot \text{ones}(N_p)$ \Comment{-1 = noise}
\State $C \gets 0$ \Comment{Current cluster index}
\For{$i \gets 1$ to $N_p$}
    \If{labels$[i] \neq$ unvisited}
        \State Continue
    \EndIf
    \State $N \gets \text{RangeQuery}(i, \varepsilon)$ \Comment{Neighbors in radius}
    \If{$|N| < $ minSamples}
        \State labels$[i] \gets -1$ \Comment{Noise}
    \Else
        \State $C \gets C + 1$
        \State \textbf{ExpandCluster}$(i, C, N, \varepsilon, \text{minSamples})$
    \EndIf
\EndFor
\State \Return labels
\end{algorithmic}
\end{algorithm}

\subsubsection{Step 4: Geodesic Dilation}

\begin{algorithm}[H]
\caption{Geodesic Dilation on Mask}
\begin{algorithmic}[1]
\Require Binary mask $H$ (from CFAR), iterations $n_{iter}=3$
\Ensure Dilated mask $H_{dil}$
\State $H_{dil} \gets H$
\State Kernel $\gets$ binary $3 \times 3$ cross
\For{$i \gets 1$ to $n_{iter}$}
    \State $H_{new} \gets \text{zeros}(H_{dil}.shape)$
    \For{$k \gets 1$ to rows($H_{dil}$) $-2$}
        \For{$n \gets 1$ to cols($H_{dil}$) $-2$}
            \State $H_{new}(k,n) = \max(H_{dil}(k-1,n), H_{dil}(k,n),$ 
            \State \hspace{3cm} $H_{dil}(k+1,n), H_{dil}(k,n-1), H_{dil}(k,n+1))$
        \EndFor
    \EndFor
    \State $H_{dil} \gets H_{new}$
\EndFor
\State \Return $H_{dil}$
\end{algorithmic}
\end{algorithm}

\subsubsection{Step 5: iSTFT with Power Threshold}

\begin{algorithm}[H]
\caption{Inverse Reconstruction (iSTFT)}
\begin{algorithmic}[1]
\Require Original STFT $X_{stft}$, dilated mask $H_{dil}$, window $w$, hop $H=256$
\Ensure Reconstructed signal $\hat{x}(n)$
\State $X_{masked} \gets X_{stft} \odot H_{dil}$ \Comment{Apply mask element-wise}
\State $N_t \gets $ cols($X_{masked}$)
\State $M \gets N_{fft}$ \Comment{Reconstructed signal length}
\State $\hat{x} \gets \text{zeros}(M)$
\For{$n \gets 0$ to $N_t - 1$}
    \State Compute iFFT: $x_n \gets \text{ifft}(X_{masked}[:, n], N_{fft})$
    \State Apply window: $x_w \gets \text{real}(x_n) \odot w$
    \State Add with overlap-add: $\hat{x}[nH : nH + N_{fft}] \mathrel{+}= x_w$
\EndFor
\State Normalize by window: $\hat{x} \gets \hat{x} / (\sum_m w[m]^2)$
\State \Return $\hat{x}$
\end{algorithmic}
\end{algorithm}

\section{Data and Validation Sources}

\subsection{IPIX Database - Maritime Radar}

An essential component of this project is the use of real data from the IPIX database (Intelligent PIxel processing for X-band radar), provided by McMaster University, Canada. This data comes from a coherent polarimetric X-band radar, installed for monitoring maritime activity.

\subsubsection{IPIX Technical Characteristics}

The IPIX radar is a high-performance system specialized in detecting objects on the sea surface in the presence of clutter noise (sea-clutter):

\begin{itemize}
    \item \textbf{RF Frequency}: 9.39 GHz (X-band) - optimal for detecting small objects
    \item \textbf{PRF (Pulse Repetition Frequency)}: 1000 Hz - allows Doppler velocity detection
    \item \textbf{Pulse length}: 200 ns - high spatial resolution
    \item \textbf{Beam width}: 0.9 degrees - excellent angular precision
    \item \textbf{Data format}: Complex I/Q (In-phase + Quadrature)
\end{itemize}

\subsubsection{What Are Complex I/Q Data?}

Unlike simple audio or synthetic signals (real), radar data are \textbf{complex}: each sample is of the form $x(t) = I(t) + j \cdot Q(t)$.

\begin{itemize}
    \item \textbf{I Component (In-phase)}: Signal projection on the cosine axis
    \item \textbf{Q Component (Quadrature)}: Projection on the sine axis (shifted by 90°)
    \item \textbf{Magnitude}: $|x(t)| = \sqrt{I^2 + Q^2}$ - echo intensity
    \item \textbf{Phase}: $\phi(t) = \arctan(Q/I)$ - Doppler information
\end{itemize}

The I/Q representation allows \textbf{detection of positive and negative Doppler frequencies}, essential for distinguishing between:
\begin{itemize}
    \item Targets approaching (positive Doppler frequency)
    \item Targets receding (negative Doppler frequency)
    \item Static clutter (Doppler frequency $\approx$ 0 Hz)
\end{itemize}

\subsubsection{Real Target Data}

For this project, we downloaded \textbf{real target files} from the McMaster IPIX database. The target is a \textbf{1-meter diameter styrofoam sphere} wrapped in wire mesh, anchored at \textbf{2660 meters} range.

\begin{table}[H]
\centering
\caption{IPIX Files with Real Targets}
\begin{tabular}{lccc}
\toprule
\textbf{File} & \textbf{Range Cell} & \textbf{Polarization} & \textbf{Sea State} \\
\midrule
\#17 (19931106\_180557) & 7 & HH & Moderate \\
\#18 (19931106\_181048) & 7 & HH & Moderate \\
\#30 (19931106\_191449) & 7 & HH & Higher \\
\#40 (19931106\_195609) & 7 & HH & Moderate \\
\bottomrule
\end{tabular}
\end{table}

\subsubsection{What Is Sea-Clutter? Interpreting IPIX Spectrograms}

Before presenting the results, it is essential to understand \textbf{what we actually see} in IPIX spectrograms. Unlike clear synthetic signals, maritime radar data contain complex physical phenomena.

\textbf{Why does the IPIX spectrogram look "strange"?}

IPIX data are \textbf{complex I/Q}, so the spectrogram is \textbf{two-sided} (negative and positive frequencies):

\begin{itemize}
    \item \textbf{Thick red line in the middle}: 0 Hz frequency (DC) - static echoes from sea surface
    \item \textbf{Yellow/Green around DC}: Active sea-clutter (waves, foam, motion)
    \item \textbf{Blue lateral areas}: Uniform weak thermal noise
    \item \textbf{"Dotted" aspect}: Energy is concentrated in time-frequency ridges, not uniform
\end{itemize}

\textbf{What is sea-clutter?}

Sea-clutter (maritime noise) represents \textbf{radar echoes from the sea surface}:

\begin{itemize}
    \item Reflections from waves, foam, water droplets
    \item Concentrated around 0 Hz frequency (small Doppler - slow motion)
    \item Energy approximately 90-95\% of cases in the $[-50, +50]$ Hz interval
    \item Has non-uniform structure - some zones are more intense (ridges)
    \item Real targets (ships, objects) appear far from DC ($\pm 100-400$ Hz)
\end{itemize}

\begin{figure}[H]
\centering
\includegraphics[width=0.95\textwidth]{../results/ipix_figures/ipix_seaclutter_explanation.png}
\caption{Visual explanation: what are sea-clutters in IPIX data? The figure shows the complete spectrogram (two-sided), zoom on the clutter region around DC (0 Hz frequency), and energy distribution across frequencies. Note that most energy (>90\%) is concentrated within $\pm 50$ Hz around DC.}
\label{fig:seaclutter_explain}
\end{figure}

\subsubsection{Why Does CFAR Not Detect "The Whole White Zone"?}

A legitimate question when looking at the results: \textit{"Why does it detect only scattered points, not the whole high-energy region?"}. The answer lies in the \textbf{locally adaptive} nature of CFAR:

\begin{itemize}
    \item \textbf{Global Threshold}: A simplistic algorithm would set a fixed threshold (e.g., "detect everything above -20 dB"). This would detect \textit{the whole white zone} --- but would also detect strong noise, generating thousands of false alarms!
    
    \item \textbf{Local Adaptive CFAR}: Compares each pixel with \textit{its neighbors} (training cells). If the pixel is "much stronger than the local neighbor average" $\rightarrow$ detected. Otherwise $\rightarrow$ ignored.
\end{itemize}

\textbf{Consequence}: Inside a uniform high-energy zone (sea-clutter), all pixels have \textit{equally strong neighbors} $\rightarrow$ CFAR doesn't detect them, because they are not "outliers" relative to the local context! CFAR detects only:
\begin{itemize}
    \item \textbf{Edges} where energy increases abruptly
    \item \textbf{Ridges} (energy crests) that exceed surrounding clutter
    \item \textbf{Real targets} that emerge from noise/clutter
\end{itemize}

That's why it seems to take "random points" --- in reality, it takes exactly the points that represent \textit{power transitions}!

This is the \textit{power} of CFAR: it dramatically reduces false alarm rate by adapting to noise. The price paid: we no longer detect "whole zones", only points that actually contain new information (abrupt changes in the spectrogram).

\section{Experimental Results}

\subsection{Experiments on Synthetic Signals}

100 Monte Carlo runs were performed for each SNR level (5, 10, 15, 20, 25, 30 dB). The synthetic signal is a nonlinear chirp according to Eq. 14 from the paper.

\begin{table}[H]
\centering
\caption{CFAR-STFT Results on Synthetic Nonlinear Chirp - 100 MC runs}
\label{tab:results_synthetic}
\begin{tabular}{ccccc}
\toprule
\textbf{SNR [dB]} & \textbf{RQF\_mean [dB]} & \textbf{RQF\_std [dB]} & \textbf{P\_detection [\%]} & \textbf{N\_runs} \\
\midrule
5 & 7.28 & 0.47 & 100.0 & 100 \\
10 & 16.81 & 0.60 & 100.0 & 100 \\
15 & 22.95 & 0.56 & 100.0 & 100 \\
20 & 26.40 & 0.51 & 100.0 & 100 \\
25 & 28.43 & 0.39 & 100.0 & 100 \\
30 & 29.17 & 0.25 & 100.0 & 100 \\
\bottomrule
\end{tabular}
\end{table}

\subsection{Experiments on Real IPIX Data with Targets}

We ran animated detection on the real target files. The figures below show the \textbf{last frames} from our detection animations:

\begin{figure}[H]
\centering
\includegraphics[width=0.95\textwidth]{../results/ipix_figures/ipix_target_17_goca_frame83.png}
\caption{GOCA-CFAR Detection on IPIX Target \#17 - Frame from animation showing active detections. The three panels show: (left) spectrogram with accumulated detections (red overlay), (center) detection heatmap, (right) current frame detections. The vertical bright line represents the floating target at a positive Doppler shift (approaching radar).}
\label{fig:goca_detection}
\end{figure}

\begin{figure}[H]
\centering
\includegraphics[width=0.95\textwidth]{../results/ipix_figures/ipix_target_17_fractal_frame83.png}
\caption{GOCA-CFAR with Fractal Boost on IPIX Target \#17 - Frame showing active detections. Fractal boost uses Hurst exponent analysis to detect targets that disrupt the self-similar structure of sea clutter, improving detection of weak targets.}
\label{fig:fractal_boost_detection}
\end{figure}

\section{Sea Clutter Adaptations}
\label{sec:adaptations}

Key modifications to adapt the paper's algorithm for real sea clutter:

\subsection{Adaptation 1: K-Distribution Instead of Gaussian}

The paper assumes \textbf{Gaussian/Rayleigh} noise statistics. Real sea clutter follows a \textbf{K-distribution} with heavy tails:
\begin{equation}
    p(x) = \frac{4}{\Gamma(\nu)} \left(\frac{\nu x^2}{2\mu}\right)^{(\nu+1)/2} K_{\nu-1}\left(\sqrt{\frac{2\nu x^2}{\mu}}\right)
\end{equation}
where $\nu$ is the shape parameter (lower = spikier) and $K_{\nu-1}$ is the modified Bessel function.

\textbf{Solution}: Estimate $\nu$ from training cell statistics ($\nu = \mu^2/(\sigma^2 - \mu^2)$) and adjust threshold multiplier accordingly. This significantly reduces false alarms on heavy-tailed clutter.

\subsection{Adaptation 2: Fractal Boost with Hurst Exponent}

CFAR occasionally missed weak targets below the adaptive threshold. Sea clutter exhibits \textbf{self-similarity} (fractal property) quantified by the \textbf{Hurst exponent}:
\begin{equation}
    E[|X(t+\tau) - X(t)|^2] \propto \tau^{2H}
\end{equation}

For sea clutter: $H \approx 0.75-0.85$ (persistent). When a target appears, it \textbf{disrupts} this structure, causing $H$ to drop below 0.6.

\textbf{Solution}: Compute $H$ via R/S analysis per frequency bin. If $H < 0.6$, flag high-power points as potential targets. Combine with CFAR: $\text{mask} = \text{CFAR} \lor (\text{Hurst anomaly} \land \text{high power})$. This improves detection of weak targets that CFAR alone would miss.

\subsection{Adaptation 3: Asymmetric DBSCAN for Vertical Lines}

Target signatures appear as \textbf{vertical lines} (many frequency bins, few time bins). Standard DBSCAN fragmented these into multiple clusters.

\textbf{Solution}: Asymmetric distance with \texttt{freq\_scale=3.0}: $d = \sqrt{\Delta t^2 + (\Delta f / 3)^2}$.

\textbf{Impact}: Single target across 50 frequency bins now correctly clusters as one detection.

\subsection{Adaptation 4: DC Component Masking}

The strong DC component (0 Hz) from stationary wave returns caused persistent false detections.

\textbf{Solution}: Mask $\pm 8$ frequency bins around DC before CFAR.

\textbf{Justification}: DC represents stationary returns; real moving targets have non-zero Doppler.

\subsection{Adaptation 5: Doppler Bandwidth Filter}

Some false alarms appeared as single-frequency narrowband detections (physically implausible).

\textbf{Solution}: Reject clusters with Doppler bandwidth $< 3$ Hz.

\section{Implementation Details}

\subsection{Parameters and Calibration}

\begin{table}[H]
\centering
\caption{Algorithm Parameters - Values Used}
\begin{tabular}{lccc}
\toprule
\textbf{Parameter} & \textbf{Value} & \textbf{Range} & \textbf{Meaning} \\
\midrule
$N_{fft}$ (window\_size) & 256 & $[128, 512]$ & FFT length \\
$H$ (hop\_size) & 32 & $[N/8, N/2]$ & Hop = 87.5\% overlap \\
$\sigma_{window}$ & 8 & [4, 16] & Window std deviation \\
$P_f$ & 0.001 & [0.0001, 0.01] & False alarm probability \\
$N_G$ (cfar\_guard) & 3 & [2, 8] & Guard cell size \\
$N_T$ (cfar\_training) & 12 & [8, 24] & Training cell size \\
$\varepsilon_{DBSCAN}$ & 8 & [4, 16] & Clustering radius \\
minSamples & 5 & [3, 10] & Minimum cluster points \\
freq\_scale & 3.0 & [2, 5] & Asymmetric DBSCAN scaling \\
dc\_mask\_bins & 8 & [4, 16] & DC bins to mask \\
min\_doppler\_bw & 3.0 Hz & [1, 10] & Minimum Doppler bandwidth \\
\bottomrule
\end{tabular}
\end{table}

\subsection{Software Dependencies}

\begin{itemize}
    \item \textbf{NumPy} $\geq$ 1.19 - matrix operations
    \item \textbf{SciPy} $\geq$ 1.5 - STFT, convolution, iFFT
    \item \textbf{matplotlib} $\geq$ 3.3 - visualizations
    \item \textbf{Pillow} - GIF frame extraction
    \item (optional) \textbf{scikit-learn} - DBSCAN reference
\end{itemize}

\section{Conclusions and Future Work}

\subsection{Main Conclusions}

The CFAR-STFT implementation demonstrates excellent performance:

\begin{enumerate}
    \item \textbf{Accuracy}: RQF = 29.17 dB @ SNR=30dB, 100\% detection
    \item \textbf{Robustness}: Consistent performance on 100 MC runs
    \item \textbf{Computational efficiency}: $\sim 75$ ms per segment (13 FPS)
    \item \textbf{Reproducibility}: Open-source code, verifiable results
    \item \textbf{Real validation}: Works on complex IPIX sea-clutter data
\end{enumerate}

\subsection{Future Development Perspectives}

\begin{itemize}
    \item \textbf{GPU acceleration}: CUDA/OpenCL implementation for real-time
    \item \textbf{Parameter optimization}: Automatic CFAR adaptation based on signal type
    \item \textbf{Real system}: Integration with operational radar
    \item \textbf{Multi-target}: Tracking and trajectory prediction
    \item \textbf{Machine learning}: Automatic parameter calibration
\end{itemize}

\subsection{Code Availability}

\textbf{GitHub Repository}: \url{https://github.com/dirgnic/Radar_Detection_STFT}

All code files, data, and results are publicly available for reproducibility and independent replication.

\begin{thebibliography}{99}

\bibitem{abratkiewicz2022}
Abratkiewicz, K. (2022).
\textit{Radar Detection-Inspired Signal Retrieval from the Short-Time Fourier Transform}.
Sensors, 22(16), 5954.
\newline \url{https://doi.org/10.3390/s22165954}

\bibitem{ipix}
S. Haykin, et al.,
``IPIX Radar Database,''
McMaster University / DREO, 1993.
\url{http://soma.ece.mcmaster.ca/ipix/}

\bibitem{ward2006}
K. D. Ward, R. J. A. Tough, S. Watts,
\textit{Sea Clutter: Scattering, the K Distribution and Radar Performance},
IET, 2006.

\bibitem{hurst1951}
H. E. Hurst,
``Long-term storage capacity of reservoirs,''
\textit{Trans. Am. Soc. Civil Eng.}, vol. 116, pp. 770-799, 1951.

\bibitem{harris1978}
Harris, F.J. (1978).
\textit{On the Use of Windows for Harmonic Analysis with the Discrete Fourier Transform}.
Proceedings of the IEEE, 66(1), 51-83.

\bibitem{ester1996}
Ester, M., Kriegel, H.-P., Sander, J., Xu, X. (1996).
\textit{A Density-Based Algorithm for Discovering Clusters in Large Spatial Databases with Noise}.
In KDD'96: Proceedings, pp. 226-231.

\bibitem{rohling1983}
H. Rohling,
``Radar CFAR thresholding in clutter and multiple target situations,''
\textit{IEEE Trans. Aerospace Electron. Syst.}, vol. 19, no. 4, 1983.

\bibitem{richards2005}
Richards, M. A. (2005).
\textit{Fundamentals of Radar Signal Processing}.
McGraw-Hill Professional.

\end{thebibliography}

\end{document}
