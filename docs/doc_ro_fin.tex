% Documentație conform template FMI (report, Times, 1.5 line spacing)
% Autori: Ingrid Corobana, Teodora Nae
% Compilare recomandată: xelatex doc_ro_fin.tex

\documentclass[12pt, a4paper]{report}

% Suport pentru diacritice și alte simboluri
\usepackage{fontspec}

% Font Times New Roman
\usepackage{times}

% Suport pentru mai multe limbi
\usepackage{polyglossia}

% Setează limba textului la română
\setdefaultlanguage{romanian}
% Am nevoie de engleză pentru rezumat
\setotherlanguages{english}

% Indentează și primul paragraf al fiecărei noi secțiuni
\SetLanguageKeys{romanian}{indentfirst=true}

% Suport pentru diferite stiluri de ghilimele
\usepackage{csquotes}

\DeclareQuoteStyle{romanian}
  {\quotedblbase}
  {\textquotedblright}
  {\guillemotleft}
  {\guillemotright}

% Setează spațiere inter-linie (doublespacing din LaTeX este echivalentul setarii 1.5 in Microsoft Word)
\usepackage{setspace}
\doublespacing

% Modificarea geometriei paginii (cu margini de 2,5 cm) 
\usepackage[margin=2.5cm]{geometry}

% Include funcțiile de grafică
\usepackage{graphicx}
% Încarcă imaginile din directorul `images`
\graphicspath{{./images/}{../results/ipix_figures/}}

% Pachete matematice
\usepackage{amsmath}
\usepackage{amssymb}

% Algoritmi
\usepackage{algorithm}
\usepackage{algpseudocode}

% Listări de cod
\usepackage{listings}

% Culori
\usepackage{xcolor}

% Linkuri interactive în PDF
\usepackage[
    colorlinks,
    linkcolor={black},
    menucolor={black},
    citecolor={black},
    urlcolor={blue}
]{hyperref}

% Tabele
\usepackage{array}
\usepackage{booktabs}
\usepackage{float}
\usepackage{caption}

% Pentru includerea PDF-urilor
\usepackage{pdfpages}

% Pachete pentru figurile complexe TikZ
\usepackage{tikz}
\usetikzlibrary{arrows.meta,positioning,fit,calc,shadows.blur,backgrounds}

% Stiluri diferite de headere și footere
\usepackage{fancyhdr}

\definecolor{stftblue}{RGB}{66, 133, 244}
\definecolor{cfarorange}{RGB}{255, 152, 0}
\definecolor{dbscangreen}{RGB}{76, 175, 80}
\definecolor{darkblue}{RGB}{25, 50, 100}
\definecolor{codegray}{RGB}{240, 240, 240}

\lstset{
    language=Python,
    basicstyle=\ttfamily\small,
    backgroundcolor=\color{codegray},
    breaklines=true,
    keywordstyle=\color{blue},
    commentstyle=\color{gray},
    stringstyle=\color{red},
    identifierstyle=\color{darkblue},
    numbers=left,
    numberstyle=\tiny\color{gray}
}

\renewcommand{\algorithmicrequire}{\textbf{Input:}}
\renewcommand{\algorithmicensure}{\textbf{Output:}}

% Metadate
\title{Analiza semnalelor radar în prezența sea clutter}
\author{Ingrid Corobana \quad Teodora Nae}

% Generează variabilele cu @
\makeatletter

% Suport pentru rezumat în două limbi
\newenvironment{abstractpage}
  {\cleardoublepage\thispagestyle{empty}}
  {\cleardoublepage}
\renewenvironment{abstract}[1]
  {\bigskip\selectlanguage{#1}%
   \begin{center}\bfseries\abstractname\end{center}}
  {\par\bigskip}

\begin{document}

% ============================================================================
% PAGINA DE TITLU - Format FMI
% ============================================================================
\begin{titlepage}
\newgeometry{left=2cm,right=2cm,top=1.5cm,bottom=1cm}
\begin{singlespace}

\begin{figure}[!htb]
    \centering
    \begin{minipage}{0.18\textwidth}
        \includegraphics[width=\linewidth]{logo-ub.png}
    \end{minipage}
    \begin{minipage}{0.55\textwidth}
        \begin{center}
            \textbf{UNIVERSITATEA DIN BUCUREȘTI}\\[0.2cm]
            \textbf{FACULTATEA DE MATEMATICĂ ȘI INFORMATICĂ}
        \end{center}
    \end{minipage}
    \begin{minipage}{0.18\textwidth}
        \includegraphics[width=\linewidth]{logo-fmi.png}
    \end{minipage}
\end{figure}

\vspace{0.3cm}
\begin{center}
\textbf{Specializarea: Informatică}
\end{center}

\vspace{0.8cm}

\begin{center}
\Large \textbf{Proiect la Procesarea Semnalelor}
\end{center}

\vspace{0.5cm}

\begin{center}
\LARGE \textbf{ANALIZA SEMNALELOR RADAR ÎN PREZENȚA SEA CLUTTER}
\end{center}

\vspace{0.3cm}

\begin{center}
\large Abordare bazată pe CFAR-STFT și experimente pe date sintetice și reale
\end{center}

\vspace{1.5cm}

\begin{center}
\large \textbf{Studenți}\\[0.1cm]
Ingrid Corobana\\
Teodora Nae
\end{center}

\vspace{0.8cm}

\begin{center}
\large \textbf{Coordonator științific}\\[0.1cm]
Prof. Dr. Cristian Rusu
\end{center}

\vspace{1.2cm}

\begin{center}
\small \textbf{Repository GitHub:} \url{https://github.com/dirgnic/Radar_Detection_STFT}
\end{center}

\vspace{0.8cm}

\begin{center}
\Large \textbf{București, 2026}
\end{center}

\end{singlespace}
\end{titlepage}

\restoregeometry
\newgeometry{margin=2.5cm}

% ============================================================================
% PAGINA DE REZUMAT
% ============================================================================
\begin{abstractpage}

\begin{abstract}{romanian}
Acest document prezintă o implementare completă a algoritmului \textbf{CFAR--STFT}, propus de \textbf{Abratkiewicz (2022)}, pentru detecția și reconstrucția semnalelor radar în prezența zgomotului și a sea clutter. Algoritmul combină \textbf{Short-Time Fourier Transform (STFT)}, \textbf{detecție adaptivă CFAR 2D}, \textbf{clustering DBSCAN} și \textbf{dilatare geodezică} pentru a extrage componenta de interes dintr-un amestec cu sea clutter.

Implementarea este validată pe date sintetice (chirp neliniar) și pe date reale (\textbf{IPIX radar}, sea clutter). Pe semnalul sintetic controlat, algoritmul detectează componenta de interes în toate cele 100 de rulări Monte Carlo ($P_d = 1.00$). \textbf{RQF} (Reconstruction Quality Factor) crește de la \textbf{7.28 dB} (SNR = 5 dB) la \textbf{29.17 dB} (SNR = 30 dB).

\textbf{Contribuția principală}: adaptăm algoritmul la sea clutter real folosind \textbf{K-distribution} (în loc de Gaussian), \textbf{îmbunătățire bazată pe proprietăți fractale (exponentul Hurst)} pentru ținte slabe și \textbf{DBSCAN asimetric} pentru clustering de semnături verticale.

\textbf{Cuvinte cheie:} CFAR, STFT, radar, sea clutter, K-distribution, DBSCAN, detecție adaptivă
\end{abstract}

\newpage

\begin{abstract}{english}
This document presents a complete implementation of the \textbf{CFAR-STFT} algorithm, proposed by \textbf{Abratkiewicz (2022)}, for detection and reconstruction of radar signals in the presence of noise and sea clutter. The algorithm combines \textbf{Short-Time Fourier Transform (STFT)}, \textbf{2D adaptive CFAR detection}, \textbf{DBSCAN clustering}, and \textbf{geodesic dilation} to extract the component of interest from a mixture with sea clutter.

The implementation is validated on synthetic data (nonlinear chirp) and real data (\textbf{IPIX radar sea clutter}). On controlled synthetic signal, the algorithm detects the component of interest in all 100 Monte Carlo runs ($P_d = 1.00$). \textbf{RQF} (Reconstruction Quality Factor) increases from \textbf{7.28 dB} (SNR = 5 dB) to \textbf{29.17 dB} (SNR = 30 dB).

\textbf{Key contribution}: we adapt the algorithm to real sea clutter using \textbf{K-distribution} (instead of Gaussian), \textbf{fractal-based enhancement (Hurst exponent)} for weak targets, and \textbf{asymmetric DBSCAN} for vertical signature clustering.

\textbf{Keywords:} CFAR, STFT, radar, sea clutter, K-distribution, DBSCAN, adaptive detection
\end{abstract}

\end{abstractpage}

% ============================================================================
% CUPRINS
% ============================================================================
\tableofcontents

% ============================================================================
% CONFIGURARE PAGINI PRINCIPALE
% ============================================================================
\cleardoublepage
\fancypagestyle{main}{
  \fancyhf{}
  \renewcommand\headrulewidth{0pt}
  \fancyhead[C]{}
  \fancyfoot[C]{\thepage}
}
\pagestyle{main}

% ============================================================================
% CAPITOLUL 1: INTRODUCERE
% ============================================================================
\chapter{Introducere}

Problema principală pe care o rezolvăm este \textbf{detecția obiectelor mici în date radar maritime}, într-un mediu complex care se schimbă constant din cauza valurilor. Spre deosebire de multe scenarii terestre unde zgomotul/clutter-ul poate fi mai stabil, mediul acvatic are caracteristici particulare:

\begin{itemize}
    \item statisticile \textbf{nu sunt bine modelate Gaussian} (apar valori extreme mai frecvent decât în distribuția normală),
    \item există \textbf{corelație temporală} (valurile creează tipare structurate),
    \item efectele Doppler duc la \textbf{extinderea spectrului} (valuri în mișcare),
    \item apar \textbf{spike-uri} vizibile în spectrograme când valurile sunt mai mari.
\end{itemize}

Metodele tradiționale adaptive de detecție \textbf{CFAR} (Constant False Alarm Rate) pot fi limitate atunci când pierd informația temporală și nu exploatează structura timp--frecvență. Abratkiewicz (2022)\footnote{Abratkiewicz, K. (2022). Radar Detection-Inspired Signal Retrieval from the Short-Time Fourier Transform. Sensors, 22(16), 5954.} propune o abordare care folosește explicit structura \textbf{time--frequency} pentru a îmbunătăți atât \textbf{detecția}, cât și \textbf{recuperarea/reconstrucția} componentelor semnalului.





% ============================================================================
% CAPITOLUL 2: STATE-OF-THE-ART (FUNDAMENTE TEORETICE)
% ============================================================================
\chapter{State-of-the-art și Fundamente}

\section{STFT - Transformata Fourier cu Timp Scurt}

\begin{equation}
X(k,n) = \sum_{m=0}^{N-1} x(m) \cdot w(m - nH) \cdot e^{-j2\pi km/N}
\end{equation}
cu $N=256$, $H=32$ (87.5\% overlap), fereastră Gaussiană $w(m) = e^{-m^2/(2\sigma^2)}$, $\sigma=8$.

\section{CFAR 2D (GOCA-CFAR)}

Detecție adaptivă cu prag local: $H(k,n) = 1$ dacă $|X(k,n)|^2 > T$, unde
\begin{equation}
T = R \cdot \hat{Z}, \quad \hat{Z} = \max(\mu_1, \mu_2, \mu_3, \mu_4), \quad R = N_T (P_f^{-1/N_T} - 1)
\end{equation}

\section{DBSCAN}

Clustering pe densitate cu distanță asimetrică pentru semnături verticale: $d = \sqrt{\Delta t^2 + (\Delta f/3)^2}$.

% ============================================================================
% CAPITOLUL 3: DESCRIEREA ALGORITMULUI
% ============================================================================
\chapter{Descrierea completă a algoritmului}

\section{Pipeline general (5 pași)}

Algoritmul complet are cinci pași:

\begin{enumerate}
    \item calcul STFT cu fereastră Gaussiană;
    \item detecție \textbf{CFAR 2D} în plan timp--frecvență;
    \item clustering DBSCAN al punctelor detectate;
    \item extinderea măștii prin \textbf{dilatare geodezică} (geodesic dilation);
    \item extragere detecții (mascare STFT).
\end{enumerate}

\subsection{Pasul 1: Calcul STFT}

Aplică formula (1) cu $N_{fft}=256$, hop=32, fereastră Gaussiană $\sigma=8$.

\subsection{Pasul 2: Detecție CFAR 2D}

Pentru fiecare bin (k,n): calculează media puterii în 4 cadrane, prag $T = R \cdot \max(\mu_1, \mu_2, \mu_3, \mu_4)$, decizie dacă $|X(k,n)|^2 > T$. Parametri: $P_f=0.001$, $N_G=3$, $N_T=12$.

\subsection{Pasul 3: Clustering DBSCAN}

Grupează punctele detectate (k,n) cu distanța asimetrică din ecuația (3). Parametri: $\varepsilon=8$, minSamples=5.

\subsection{Pasul 4: Dilatare geodezică}

Aplică dilatare cu kernel cruce (3×3) de 3 ori: \texttt{for i in range(3): H\_dil = scipy.ndimage.maximum\_filter(H\_dil, footprint=cross)}. Expandează detecțiile pentru a conecta punctele apropiate.

\subsection{Pasul 5: Extragere detecții}

Aplică masca pe STFT: \texttt{X\_masked = X\_stft \(\odot\) H\_dil}, apoi extrage binarele detectate.

% ============================================================================
% CAPITOLUL 4: DATE ȘI SURSE DE VALIDARE
% ============================================================================
\chapter{Date și surse de validare}

\section{Baza de date IPIX}

IPIX (McMaster University): radar X-band coerent, $f_{RF}=9.39$ GHz, PRF=1000 Hz, date complexe I/Q. Ținta: sferă 1m la 2660m. Ținte validate: \#17, \#18, \#30, \#40.

\section{Experimente pe semnale sintetice}

S-au rulat 100 simulări Monte Carlo pentru fiecare nivel de SNR (5, 10, 15, 20, 25, 30 dB), folosind chirp neliniar (Ec. 14). Rata de detecție: 100\% în toate rulările.

Metrica RQF:
\begin{equation}
\text{RQF} = 10 \log_{10}\left(\frac{\sum_n |x[n]|^2}{\sum_n |x[n] - \hat{x}[n]|^2}\right) \text{ [dB]}
\end{equation}

\begin{table}[H]
\centering
\caption{Rezultate CFAR-STFT pe chirp neliniar sintetic -- 100 rulări MC}
\label{tab:results_synthetic}
\begin{tabular}{ccccc}
\toprule
\textbf{SNR [dB]} & \textbf{RQF\_mean [dB]} & \textbf{RQF\_std [dB]} & \textbf{P\_detecție [\%]} & \textbf{N\_rulări} \\
\midrule
5 & 7.28 & 0.47 & 100.0 & 100 \\
10 & 16.81 & 0.60 & 100.0 & 100 \\
15 & 22.95 & 0.56 & 100.0 & 100 \\
20 & 26.40 & 0.51 & 100.0 & 100 \\
25 & 28.43 & 0.39 & 100.0 & 100 \\
30 & 29.17 & 0.25 & 100.0 & 100 \\
\bottomrule
\end{tabular}
\end{table}

\section{Experimente pe IPIX cu ținte reale}

Am rulat detecție animată pe fișierele cu ținte reale. Figurile de mai jos arată cadre din animațiile de detecție:

\begin{figure}[H]
\centering
\includegraphics[width=0.95\textwidth]{../results/ipix_figures/ipix_target_17_goca_frame83.pdf}
\caption{Detecție GOCA-CFAR pe IPIX Target \#17 -- Cadru din animație arătând detecții active. Cele trei panouri arată: (stânga) spectrograma cu detecții acumulate (overlay roșu), (centru) heatmap de detecție, (dreapta) detecțiile cadrului curent. Linia verticală luminoasă reprezintă ținta plutitoare la Doppler pozitiv (se apropie de radar).}
\label{fig:goca_detection}
\end{figure}

\begin{figure}[H]
\centering
\includegraphics[width=0.95\textwidth]{../results/ipix_figures/ipix_target_17_fractal_frame83.pdf}
\caption{GOCA-CFAR cu Fractal Boost pe IPIX Target \#17 -- Cadru arătând detecții active. Fractal boost folosește analiza exponentului Hurst pentru a detecta ținte care perturbă structura self-similar a sea clutter, îmbunătățind detecția țintelor slabe.}
\label{fig:fractal_boost_detection}
\end{figure}

% ============================================================================
% CAPITOLUL 6: ADAPTĂRI PENTRU SEA CLUTTER
% ============================================================================
\chapter{Adaptări pentru sea clutter}
\label{chap:adaptations}

Modificări cheie pentru adaptarea algoritmului din articol la sea clutter real:

\section{Adaptarea 1: K-distribution}

\begin{equation}
p(x) = \frac{4}{\Gamma(\nu)} \left(\frac{\nu x^2}{2\mu}\right)^{(\nu+1)/2} K_{\nu-1}\left(\sqrt{\frac{2\nu x^2}{\mu}}\right)
\end{equation}

Sea clutter urmează K-distribution (cozi mai grele decât Gaussian). Estimare: $\nu = \mu^2 / (\sigma^2 - \mu^2)$. Ajustare prag pentru reducerea alarmelor false.

\section{Adaptarea 2: Exponentul Hurst}

\begin{equation}
\mathbb{E}\left[|X(t+\tau) - X(t)|^2\right] \propto \tau^{2H}
\end{equation}

Sea clutter: $H \approx 0.75$--$0.85$; ținte: $H < 0.6$. Mască combinată: $\text{CFAR} \lor (\text{Hurst anomaly} \land \text{putere mare})$ pentru detecție ținte slabe.

\section{Adaptarea 3: DBSCAN asimetric pentru semnături verticale}

Țintele apar ca \textbf{linii verticale} (multe frecvențe, puține momente). DBSCAN standard fragmentează acestea în clustere multiple.

\textbf{Soluție}: Distanță asimetrică }

Distanță asimetrică: $d = \sqrt{\Delta t^2 + (\Delta f / 3)^2}$ cu \texttt{freq\_scale=3.0}. Ținte pe 50 bin-uri frecvență $\rightarrow$ 1valuri.

\textbf{Soluție}: Mascăm $\pm DC}

Mască $\pm 8$ bin-uri frecvență în jurul DC înainte de CFAR (returnări staționare non-ținte)

\textbf{Soluție}: Respingem cDoppler}

Respingem clustere cu bandwidth Doppler $< 3$ Hz (detecții fizic implauzibile)============
\chapter{Detalii de implementare}

\section{Parametri și calibrare}

\begin{table}[H]
\centering
\caption{Parametrii algoritmului -- valori utilizate}
\begin{tabular}{lccc}
\toprule
\textbf{Parametru} & \textbf{Valoare} & \textbf{Interval} & \textbf{Semnificație} \\
\midrule
$N_{fft}$ (window\_size) & 256 & $[128, 512]$ & Lungime FFT \\
$H$ (hop\_size) & 32 & $[N/8, N/2]$ & Hop = 87.5\% overlap \\
$\sigma_{window}$ & 8 & [4, 16] & Deviația std. a ferestrei \\
$P_{fa}$ & 0.001 & [0.0001, 0.01] & Probabilitate alarmă falsă \\
$N_G$ (cfar\_guard) & 3 & [2, 8] & Mărime celule guard \\
$N_T$ (cfar\_training) & 12 & [8, 24] & Mărime celule training \\
$\varepsilon_{DBSCAN}$ & 8 & [4, 16] & Raza de clustering \\
minSamples & 5 & [3, 10] & Puncte minime per cluster \\
freq\_scale & 3.0 & [2, 5] & Scalare DBSCAN asimetric \\
dc\_mask\_bins & 8 & [4, 16] & Bin-uri DC de mascat \\
min\_doppler\_bw & 3.0 Hz & [1, 10] & Lățime Doppler minimă \\
\bottomrule
\end{tabular}
\end{table}

\section{Interpretare Doppler}

Conversie Doppler $\rightarrow$ viteză radială:
\begin{equation}
v_r = \frac{f_d \cdot c}{2 f_{RF}}
\end{equation}

Pentru IPIX ($f_{RF} = 9.39$ GHz): $f_d = +100$ Hz $\rightarrow$ $v_r \approx +1.6$ m/s; $f_d = -100$ Hz $\rightarrow$ $v_r \approx -1.6$ m/s.

Viteza maximă neambiguă:
\begin{equation}
v_{max} = \frac{PRF \cdot c}{4 f_{RF}} \approx \pm 8 \text{ m/s}
\end{equation}

% ============================================================================
% CAPITOLUL 8: CONCLUZII
% ============================================================================
\chapter{Concluzii și direcții viitoare}

\section{Concluzii}

Implementarea CFAR-STFT demonstrează: (1) RQF = 29.17 dB @ SNR=30dB cu 100\% detecție pe sintetice; (2) performance consistentă pe 100 MC runs; (3) validare pe date IPIX reale cu sea clutter complex; (4) reproductibilitate via GitHub.

\section{Direcții viitoare}

GPU acceleration, auto-calibration, sistem operațional, multi-target tracking, ML-based parameter adaptation.

\section{Cod}

GitHub: \url{https://github.com/dirgnic/Radar_Detection_STFT} (reproductibilitate, replicare independentă).

\begin{thebibliography}{99}

\bibitem{abratkiewicz2022}
Abratkiewicz, K. (2022).
\textit{Radar Detection-Inspired Signal Retrieval from the Short-Time Fourier Transform}.
Sensors, 22(16), 5954.
\newline \url{https://doi.org/10.3390/s22165954}

\bibitem{ipix}
S. Haykin, et al.,
``IPIX Radar Database,''
McMaster University / DREO, 1993.
\url{http://soma.ece.mcmaster.ca/ipix/}

\bibitem{ward2006}
K. D. Ward, R. J. A. Tough, S. Watts,
\textit{Sea Clutter: Scattering, the K Distribution and Radar Performance},
IET, 2006.

\bibitem{hurst1951}
H. E. Hurst,
``Long-term storage capacity of reservoirs,''
\textit{Trans. Am. Soc. Civil Eng.}, vol. 116, pp. 770-799, 1951.

\bibitem{harris1978}
Harris, F.J. (1978).
\textit{On the Use of Windows for Harmonic Analysis with the Discrete Fourier Transform}.
Proceedings of the IEEE, 66(1), 51-83.

\bibitem{ester1996}
Ester, M., Kriegel, H.-P., Sander, J., Xu, X. (1996).
\textit{A Density-Based Algorithm for Discovering Clusters in Large Spatial Databases with Noise}.
In KDD'96: Proceedings, pp. 226-231.

\bibitem{rohling1983}
H. Rohling,
``Radar CFAR thresholding in clutter and multiple target situations,''
\textit{IEEE Trans. Aerospace Electron. Syst.}, vol. 19, no. 4, 1983.

\bibitem{richards2005}
Richards, M. A. (2005).
\textit{Fundamentals of Radar Signal Processing}.
McGraw-Hill Professional.

\end{thebibliography}

% ============================================================================
% ANEXĂ: DIAGRAME PIPELINE
% ============================================================================
\appendix
\chapter{Diagrame detaliate ale pipeline-ului CFAR-STFT}

Planul timp--frecvență și componentele algoritmului sunt ilustrate în figura \ref{fig:pipeline-pdf}.

\begin{figure}[H]
\centering
\includepdf[pages=-, scale=0.75]{pipeline_diagrams.pdf}
\caption{Diagrame detaliate: (1) Pipeline principal, (2) CFAR 2D, (3) Configurația experimentală, (4) Procesare TF, (5) Doppler, (6) Parametri, (7) RQF metric}
\label{fig:pipeline-pdf}
\end{figure}

\end{document}
