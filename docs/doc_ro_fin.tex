% Documentație conform template FMI (report, Times, 1.5 line spacing)
% Autori: Ingrid Corobana, Teodora Nae
% Compilare recomandată: xelatex doc_ro_fin.tex

\documentclass[11pt, a4paper]{report}

% Suport pentru diacritice și alte simboluri
\usepackage{fontspec}

% Font Times New Roman
\usepackage{times}

% Suport pentru mai multe limbi
\usepackage{polyglossia}

% Setează limba textului la română
\setdefaultlanguage{romanian}
% Am nevoie de engleză pentru rezumat
\setotherlanguages{english}

% Indentează și primul paragraf al fiecărei noi secțiuni
\SetLanguageKeys{romanian}{indentfirst=true}

% Suport pentru diferite stiluri de ghilimele
\usepackage{csquotes}

\DeclareQuoteStyle{romanian}
  {\quotedblbase}
  {\textquotedblright}
  {\guillemotleft}
  {\guillemotright}

% Setează spațiere inter-linie la single (compress)
\usepackage{setspace}
\singlespacing

% Modificarea geometriei paginii (cu margini mai mici - 0.5 inch)
\usepackage[margin=0.5in]{geometry}

% Include funcțiile de grafică
\usepackage{graphicx}
% Încarcă imaginile din directoarele specificate
\graphicspath{{./images/}{../results/ipix_figures/}{../results/evaluation/}}

% Pachete matematice
\usepackage{amsmath}
\usepackage{amssymb}

% Algoritmi
\usepackage{algorithm}
\usepackage{algpseudocode}

% Listări de cod
\usepackage{listings}

% Culori
\usepackage{xcolor}

% Linkuri interactive în PDF
\usepackage[
    colorlinks,
    linkcolor={black},
    menucolor={black},
    citecolor={black},
    urlcolor={blue}
]{hyperref}

% Tabele
\usepackage{array}
\usepackage{booktabs}
\usepackage{float}
\usepackage{caption}
\usepackage{multirow}

% Reduce caption spacing
\captionsetup{skip=3pt}

% Pentru includerea PDF-urilor
\usepackage{pdfpages}

% Pachete pentru figurile complexe TikZ
\usepackage{tikz}
\usetikzlibrary{arrows.meta,positioning,fit,calc,shadows.blur,backgrounds}

% Reduce spacing around sections (aggressive but balanced)
\usepackage{titlesec}
\titlespacing{\chapter}{0pt}{2pt}{1pt}
\titlespacing{\section}{0pt}{2pt}{1pt}
\titlespacing{\subsection}{0pt}{1pt}{0pt}

% Reduce spacing in lists
\usepackage{enumitem}
\setlist{nosep}

% Reduce table row height
\renewcommand{\arraystretch}{0.75}

% Reduce space around equations
\setlength{\abovedisplayskip}{3pt}
\setlength{\belowdisplayskip}{3pt}
\setlength{\abovedisplayshortskip}{2pt}
\setlength{\belowdisplayshortskip}{2pt}

% Reduce space around floats (figures, tables)
\setlength{\floatsep}{3pt}
\setlength{\textfloatsep}{3pt}
\setlength{\intextsep}{2pt}

% Reduce paragraph spacing (but keep readable)
\setlength{\parskip}{4pt plus 2pt}

% Stiluri diferite de headere și footere
\usepackage{fancyhdr}

\definecolor{stftblue}{RGB}{66, 133, 244}
\definecolor{cfarorange}{RGB}{255, 152, 0}
\definecolor{dbscangreen}{RGB}{76, 175, 80}
\definecolor{maskpurple}{RGB}{156, 39, 176}
\definecolor{istftred}{RGB}{244, 67, 54}
\definecolor{signalgray}{RGB}{96, 96, 96}
\definecolor{darkblue}{RGB}{25, 50, 100}
\definecolor{codegray}{RGB}{240, 240, 240}

\lstset{
    language=Python,
    basicstyle=\ttfamily\small,
    backgroundcolor=\color{codegray},
    breaklines=true,
    keywordstyle=\color{blue},
    commentstyle=\color{gray},
    stringstyle=\color{red},
    identifierstyle=\color{darkblue},
    numbers=left,
    numberstyle=\tiny\color{gray}
}

\renewcommand{\algorithmicrequire}{\textbf{Input:}}
\renewcommand{\algorithmicensure}{\textbf{Output:}}

% Metadate
\title{Analiza semnalelor radar în prezența sea clutter}
\author{Ingrid Corobana \quad Teodora Nae}

% Generează variabilele cu @
\makeatletter

% Suport pentru rezumat în două limbi
\newenvironment{abstractpage}
  {\cleardoublepage\thispagestyle{empty}}
  {\cleardoublepage}
\renewenvironment{abstract}[1]
  {\bigskip\selectlanguage{#1}%
   \begin{center}\bfseries\abstractname\end{center}}
  {\par\bigskip}

\begin{document}

% ============================================================================
% PAGINA DE TITLU - Format FMI
% ============================================================================
\begin{titlepage}
\newgeometry{left=2cm,right=2cm,top=1.5cm,bottom=1cm}
\begin{singlespace}

\begin{figure}[!htb]
    \centering
    \begin{minipage}{0.18\textwidth}
        \includegraphics[width=\linewidth]{logo-ub.png}
    \end{minipage}
    \begin{minipage}{0.55\textwidth}
        \begin{center}
            \textbf{UNIVERSITATEA DIN BUCUREȘTI}\\[0.2cm]
            \textbf{FACULTATEA DE MATEMATICĂ ȘI INFORMATICĂ}
        \end{center}
    \end{minipage}
    \begin{minipage}{0.18\textwidth}
        \includegraphics[width=\linewidth]{logo-fmi.png}
    \end{minipage}
\end{figure}

\vspace{0.3cm}
\begin{center}
\textbf{Specializarea: Informatică}
\end{center}

\vspace{0.8cm}

\begin{center}
\Large \textbf{Proiect la Procesarea Semnalelor}
\end{center}

\vspace{0.5cm}

\begin{center}
\LARGE \textbf{ANALIZA SEMNALELOR RADAR ÎN PREZENȚA SEA CLUTTER}
\end{center}

\vspace{0.3cm}

\begin{center}
\large Abordare bazată pe CFAR-STFT și experimente pe date sintetice și reale
\end{center}

\vspace{1.5cm}

\begin{center}
\large \textbf{Studenți}\\[0.1cm]
Ingrid Corobana\\
Teodora Nae
\end{center}

\vspace{0.8cm}

\begin{center}
\large \textbf{Coordonator științific}\\[0.1cm]
Prof. Dr. Cristian Rusu
\end{center}

\vspace{1.2cm}

\begin{center}
\small \textbf{Repository GitHub:} \url{https://github.com/dirgnic/Radar_Detection_STFT}
\end{center}

\vspace{0.8cm}

\begin{center}
\Large \textbf{București, 2026}
\end{center}

\end{singlespace}
\end{titlepage}

\restoregeometry
\newgeometry{margin=2.5cm}

% ============================================================================
% PAGINA DE REZUMAT
% ============================================================================
\begin{abstractpage}

\begin{abstract}{romanian}
Acest document prezintă o implementare completă a algoritmului \textbf{CFAR--STFT}, propus de \textbf{\mbox{Abratkiewicz~(2022)}} \cite{abratkiewicz2022}, pentru detecția și reconstrucția semnalelor radar în prezența zgomotului și a \textit{sea clutter}-ului (zgomot de fond produs de suprafața mării). Pornind de la o reprezentare \textbf{timp--frecvență (STFT)}, aplicăm un prag adaptiv local (\textbf{CFAR 2D}) pentru a obține o matrice de detecție rară, pe care o stabilizăm prin grupare (\textbf{DBSCAN}) și conectare morfologică (\textbf{dilatare geodezică}), astfel încât reconstrucția să rețină componenta de interes.

Implementarea este validată pe date sintetice (chirp neliniar) și pe date reale (\textbf{IPIX radar}, \textit{``sea clutter''} (date reale)) \cite{ipix}. Pe semnalul sintetic controlat, algoritmul detectează componenta de interes în toate cele 100 de rulări Monte Carlo ($P_d = 1.00$). \textbf{RQF} (Reconstruction Quality Factor) crește de la \textbf{7.28 dB} (SNR = 5 dB) la \textbf{29.17 dB} (SNR = 30 dB).

\textbf{Contribuția principală}: adaptăm algoritmul pe date reale folosind \textbf{distribuția K} (în loc de Gaussian) \cite{ward2006}, o \textbf{îmbunătățire bazată pe proprietăți fractale (exponentul Hurst)} pentru target-uri slabe \cite{hurst1951} și \textbf{DBSCAN cu metrică anizotropică} pentru clustering de semnături cvasi-verticale. De asemenea, realizăm o analiză comparativă cu algoritmi alternativi: \textbf{CA-CFAR + HDBSCAN} și \textbf{separare geometrică prin triangulare Delaunay}.

\textbf{Cuvinte cheie:} CFAR, STFT, radar, sea clutter, K-distribution, DBSCAN, detecție adaptivă
\end{abstract}

\newpage

\begin{abstract}{english}
This document presents a complete implementation of the \textbf{CFAR-STFT} algorithm, proposed by \textbf{\mbox{Abratkiewicz~(2022)}} \cite{abratkiewicz2022}, for detection and reconstruction of radar signals in the presence of noise and sea clutter (sea-surface backscatter). Starting from a \textbf{time--frequency (STFT)} representation, we apply a local adaptive threshold (\textbf{2D CFAR}) to obtain a sparse mask, which is then stabilized via grouping (\textbf{DBSCAN}) and morphological linking (\textbf{geodesic dilation}) so that the reconstruction preserves the component of interest.

The implementation is validated on synthetic data (nonlinear chirp) and real data (\textbf{IPIX radar}, \textit{``sea clutter''} real measurements) \cite{ipix}. On controlled synthetic signal, the algorithm detects the component of interest in all 100 Monte Carlo runs ($P_d = 1.00$). \textbf{RQF} (Reconstruction Quality Factor) increases from \textbf{7.28 dB} (SNR = 5 dB) to \textbf{29.17 dB} (SNR = 30 dB).

\textbf{Key contribution}: we adapt the algorithm to real sea clutter using a \textbf{K-distribution} (instead of Gaussian) \cite{ward2006}, a \textbf{fractal-based enhancement (Hurst exponent)} for weak targets \cite{hurst1951}, and an \textbf{anisotropic DBSCAN metric} for clustering quasi-vertical signatures. We also include a comparative analysis against alternative pipelines: \textbf{CA-CFAR + HDBSCAN} and \textbf{Delaunay-triangulation-based separation}.

\textbf{Keywords:} CFAR, STFT, radar, sea clutter, K-distribution, DBSCAN, adaptive detection
\end{abstract}

\end{abstractpage}

% ============================================================================
% CUPRINS
% ============================================================================
\tableofcontents

% ============================================================================
% CONFIGURARE PAGINI PRINCIPALE
% ============================================================================
\cleardoublepage
\fancypagestyle{main}{
  \fancyhf{}
  \renewcommand\headrulewidth{0pt}
  \fancyhead[C]{}
  \fancyfoot[C]{\thepage}
}
\pagestyle{main}

% ============================================================================
% CAPITOLUL 1: INTRODUCERE
% ============================================================================
\chapter{Introducere}

Problema principală abordată este \textbf{detecția obiectelor mici în date radar maritime}, într-un mediu complex. Mediul acvatic are următoarele caracteristici particulare: (1) statisticile \textbf{nu sunt bine modelate de o distribuție Gaussiană} (valorile extreme sunt frecvente), (2) există \textbf{corelație temporală} (valuri cu tipare structurate), (3) efectele Doppler \textbf{extind spectrul} (valuri în mișcare), (4) apar \textbf{vârfuri de energie (spikes)} vizibile când valurile sunt mari.

Metodele tradiționale adaptive de detecție \textbf{CFAR} (Constant False Alarm Rate) pot fi limitate atunci când pierd informația temporală și nu exploatează structura timp--frecvență. \mbox{Abratkiewicz~(2022)}\footnote{Abratkiewicz, K. (2022). Radar Detection-Inspired Signal Retrieval from the Short-Time Fourier Transform. Sensors, 22(16), 5954.} propune o abordare care folosește explicit structura \textbf{timp--frecvență} pentru a îmbunătăți atât \textbf{detecția}, cât și \textbf{recuperarea/reconstrucția} componentelor semnalului.





% ============================================================================
% CAPITOLUL 2: STADIUL ACTUAL (FUNDAMENTE TEORETICE)
% ============================================================================
% ============================================================================
% CAPITOLUL 2: DESCRIEREA COMPLETĂ A ALGORITMULUI
% ============================================================================
\chapter{Descrierea completă a algoritmului}

\vspace{4pt}
\section{Fundamente teoretice și STFT}

\begin{samepage}
\begin{table}[H]
\centering
\scriptsize
\setlength{\tabcolsep}{2pt}
\renewcommand{\arraystretch}{1.10}
\begin{tabular}{@{} >{\raggedright\arraybackslash}p{2.4cm} >{\raggedright\arraybackslash}p{6.2cm} >{\raggedright\arraybackslash}p{5.4cm} @{}}
\toprule
\textbf{Componentă} & \textbf{Model matematic} & \textbf{Rol în algoritm} \\ \midrule

STFT (Short-Time Fourier Transform) &
\begin{tabular}[t]{@{}>{\raggedright\arraybackslash}p{\linewidth}@{}}
$\begin{aligned}
X(k,n) &= \sum_{m=0}^{N-1} x(m)\, \\
&\quad w(m-nH)\,e^{-j2\pi km/N}
\end{aligned}$\\
\textit{Param.:} $N=256$, $H=32$, fereastră Gauss, $\sigma=8$
\end{tabular} &
Transformă semnalul în plan timp--frecvență; spectrograma $|X(k,n)|^2$ este input-ul pentru CFAR și clustering. \\ \midrule

CFAR 2D (GOCA-CFAR) &
\begin{tabular}[t]{@{}>{\raggedright\arraybackslash}p{\linewidth}@{}}
$H(k,n)=\mathbf{1}\{|X(k,n)|^2>T(k,n)\}$\\
$T(k,n)=\alpha\,\hat{Z}$\\
$\hat{Z}=\max(\mu_1,\mu_2,\mu_3,\mu_4)$\\
\textit{unde} $\alpha$ depinde de $P_{fa}$ și $N_T$
\end{tabular} &
Detectează candidații de target pe baza unui prag adaptiv local, folosind 4 cadrane de training (colțuri); controlează rata de alarmă falsă. \\ \midrule

DBSCAN -- clustering pe densitate &
\begin{tabular}[t]{@{}>{\raggedright\arraybackslash}p{\linewidth}@{}}
$d=\sqrt{(s_t\,\Delta t)^2 + (\Delta f/s_f)^2}$\\
\textit{Anizotropic} $\Rightarrow$ scalare diferită pe axe; \textit{aici} $s_f=3$, $s_t=1.5$\\
\textit{Notă:} $\Delta f,\Delta t$ sunt măsurate în unități normalizate la rezoluția STFT (bin-uri)
\end{tabular} &
Grupează punctele detectate $(k,n)$ în clustere coerente (target-uri radar) păstrând structura aproape verticală. \\

\bottomrule
\end{tabular}
\caption{Fundamente teoretice: STFT, detecția CFAR 2D și clusteringul DBSCAN în cadrul algoritmului CFAR-STFT.}
\end{table}
\vspace{2pt}

\vspace{2pt}
\section{Pipeline general (5 pași)}

% Punem pașii în paralel cu schema (Figura 2.1) ca să nu ocupe o pagină întreagă.
\noindent
\vspace{-2pt}
\begin{minipage}[t]{0.42\textwidth}
\vspace{0pt}
\centering
\resizebox{0.80\linewidth}{!}{%
\begin{tikzpicture}[
    node distance=0.75cm and 0.75cm,
    box/.style={rectangle, draw, rounded corners, minimum width=1.55cm, minimum height=0.6cm, align=center, font=\scriptsize\bfseries},
    arrow/.style={-{Stealth[length=2mm]}, thick},
    label/.style={font=\tiny, align=center},
    eqlabel/.style={font=\tiny, text=gray}
]

% Două rânduri (mai lizibil, tip "serpentină"): rândul 1 stânga->dreapta, rândul 2 dreapta->stânga
\node[box, fill=signalgray!20] (input) {Input\\$x[n]$};
\node[box, fill=stftblue!30, right=of input] (stft) {STFT\\$\sigma=8$};
\node[box, fill=cfarorange!30, right=of stft] (cfar) {GOCA-CFAR\\$P_{fa}=0.001$};

\node[box, fill=dbscangreen!30, below=of cfar] (dbscan) {DBSCAN};
\node[box, fill=maskpurple!30, left=of dbscan] (mask) {Mask\\+ dilation};
\node[box, fill=istftred!30, left=of mask] (output) {Detecții};

\draw[arrow] (input) -- (stft);
\draw[arrow] (stft) -- (cfar);
\draw[arrow] (cfar) -- (dbscan);
\draw[arrow] (dbscan) -- (mask);
\draw[arrow] (mask) -- (output);

\end{tikzpicture}%
} % resizebox
\end{minipage}\hfill%
\begin{minipage}[t]{0.57\textwidth}
\vspace{0pt}
\footnotesize
\raggedright
\setlength{\emergencystretch}{1em}
\textbf{Algoritmul complet constă din cinci etape:}
\begin{enumerate}[leftmargin=*,label=\textbf{(\arabic*)},topsep=0pt,itemsep=0pt,parsep=0pt]
\item \textbf{STFT}: fereastră Gaussiană; $N_{fft}=256$, $H=32$, $\sigma=8$.
\item \textbf{GOCA-CFAR 2D}:\\
prag $T=\alpha\cdot\max(\mu_1,\ldots,\mu_4)$;\\
decizie $|X(k,n)|^2 > T$.
\item \textbf{DBSCAN}:\\
distanță anizotropică în unități de bin: $d=\sqrt{(s_t\Delta t)^2 + (\Delta f/s_f)^2}$; tipic $\varepsilon=8$, \texttt{minSamples}=5.
\item \textbf{Dilatare geodezică}:\\
dilatare geodezică iterativă cu conectivitate 4, limitată de o mască de energie joasă (până la convergență / max. iterații).
\item \textbf{Extragere}:\\
$X_{\text{masked}} = X_{\text{STFT}}\odot H_{\text{dil}}$.
\end{enumerate}
\end{minipage}
\par\vspace{8pt}
{\captionsetup{skip=1pt,font=scriptsize}\captionof{figure}{Pipeline-ul complet al algoritmului CFAR-STFT pentru detecția componentelor din planul timp--frecvență.}\label{fig:pipeline}}
\vspace{-4pt}
\end{samepage}

\subsection{Pasul 2: Structura CFAR 2D}

\begin{figure}[H]
\centering
\begin{tikzpicture}[scale=0.65]
    % Grid background
    \draw[step=1, gray!30, very thin] (-5.5,-5.5) grid (5.5,5.5);

    % GOCA-CFAR: mediile mu_k se calculeaza DOAR pe celulele de training,
    % EXCLUZAND zona de garda (guard) si CUT. Pentru claritate, desenam
    % cele 4 regiuni de training ca "L-shape" (outer quadrant minus guard).
    \path[fill=blue!18, even odd rule] (-5,0.5) rectangle (-0.5,5) (-3,0.5) rectangle (-0.5,3);
    \path[fill=green!18, even odd rule] (0.5,0.5) rectangle (5,5) (0.5,0.5) rectangle (3,3);
    \path[fill=purple!16, even odd rule] (-5,-5) rectangle (-0.5,-0.5) (-3,-3) rectangle (-0.5,-0.5);
    \path[fill=orange!18, even odd rule] (0.5,-5) rectangle (5,-0.5) (0.5,-3) rectangle (3,-0.5);

    % Guard cells (middle ring) - nu intra in mu_k
    \fill[yellow!40] (-3,-3) rectangle (3,3);

    % CUT (center)
    \fill[red!60] (-0.5,-0.5) rectangle (0.5,0.5);

    % Labels pentru regiuni
    \node at (0,0) {\textbf{CUT}};
    \node at (0,2) {\small Guard};
    \node at (0,4) {\small Training};

    % Dimensiuni
    \draw[{Stealth}-{Stealth}, thick] (-5,-6) -- (5,-6);
    \node at (0,-6.5) {$2(N_G + N_T) + 1$};
    
    \draw[{Stealth}-{Stealth}, thick] (-3,-5.3) -- (3,-5.3);
    \node at (0,-5.8) {\scriptsize $2N_G + 1$};
    
    % GOCA training regions (mu_k) - L-shape (training only, fara guard + CUT)
    \draw[thick, blue, dashed] (-5,0.5) rectangle (-3,3);
    \draw[thick, blue, dashed] (-5,3) rectangle (-0.5,5);
    \draw[thick, green!70!black, dashed] (3,0.5) rectangle (5,3);
    \draw[thick, green!70!black, dashed] (0.5,3) rectangle (5,5);
    \draw[thick, purple, dashed] (-5,-3) rectangle (-3,-0.5);
    \draw[thick, purple, dashed] (-5,-5) rectangle (-0.5,-3);
    \draw[thick, orange!80!black, dashed] (3,-3) rectangle (5,-0.5);
    \draw[thick, orange!80!black, dashed] (0.5,-5) rectangle (5,-3);
    
    % Label-urile mu_k sunt in zona de training (nu in guard)
    \node[blue, font=\scriptsize] at (-4.0,4.0) {$\mu_1$};
    \node[green!70!black, font=\scriptsize] at (4.0,4.0) {$\mu_2$};
    \node[purple, font=\scriptsize] at (-4.0,-4.0) {$\mu_3$};
    \node[orange!80!black, font=\scriptsize] at (4.0,-4.0) {$\mu_4$};
    
\end{tikzpicture}
\hspace{1cm}
\begin{tikzpicture}[scale=0.85]
	    % Formula box (wider so the main equation stays on a single line)
	    \node[draw, rounded corners, fill=gray!10, text width=7.2cm, minimum width=7.2cm, minimum height=4cm, align=left] (formulas) {
		        \begin{tabular}{@{}l@{}}
	        \textbf{GOCA-CFAR:}\\[0.2em]
	        {\small$\begin{aligned}
	        \hat{Z} &= \max(\mu_1,\mu_2,\mu_3,\mu_4) \\
	        T &= \alpha\cdot\hat{Z}
	        \end{aligned}$}\\[0.4em]
	        {\scriptsize Rayleigh (CA, aprox.): $\alpha=N_T(P_{fa}^{-1/N_T}-1)$}\\
	        {\scriptsize K-dist.: $\alpha=q_K(P_{fa};a_K,b_K)/b_K$}\\[0.5em]
		        \textbf{Decizie:}\\[0.2em]
	        $P_{\text{CUT}} \ge T \Rightarrow$ \textcolor{green!60!black}{Det}\\
	        $P_{\text{CUT}} < T \Rightarrow$ \textcolor{red}{Zgomot}\\
	        {\scriptsize unde $P_{\text{CUT}}=|X(k,n)|^2$ (power STFT)}
	        \end{tabular}
	    };
\end{tikzpicture}
\caption{Structura celulelor GOCA-CFAR 2D: CUT (roșu), Guard (galben), Training (colorat pe 4 cadrane). Mediile $\mu_1,\dots,\mu_4$ se calculează \textbf{doar} pe celulele de training (excluzând zona de guard și CUT), iar GOCA folosește maximul pentru adaptare locală.}
\label{fig:cfar}
\end{figure}

\noindent\textit{Notă:} pentru a evita contaminarea estimării de fond de către „ridge”-uri/sidelobe-uri ale unui target puternic (energie care se scurge pe axele timp/frecvență sau range/Doppler), se exclude o „cruce” de celule pe axele verticală/orizontală dintre cadrane (buffer alb); în implementarea de față aceasta este realizată implicit prin definirea cadranelor în colțuri, separate de zona de guard.

% ============================================================================
% CAPITOLUL 3: DATE ȘI SURSE DE VALIDARE
% ============================================================================
\chapter{Date și surse de validare}
\vspace{6pt}

% Mai mult "aer" intre (sub)capitole (doar in Capitolul 3).
% Crestem moderat distanta, fara sa impinga continutul pe pagini noi.
\titlespacing{\section}{0pt}{6pt}{2pt}
\titlespacing{\subsection}{0pt}{6pt}{1pt}

\section{Baza de date IPIX}

IPIX (McMaster University) \cite{ipix}: radar X-band ($f_{RF}=9.39$ GHz, PRF=1000 Hz), date complexe I/Q. Target-uri reale (sferă 1m la 2660m): \#17, \#18, \#30, \#40.

\section{Experimente pe semnale sintetice}

S-au rulat 100 de simulări Monte Carlo pentru fiecare nivel SNR, folosind un chirp neliniar, cu rată de detecție 100\%.

Metrica RQF:
\begin{equation}
\text{RQF} = 10 \log_{10}\left(\frac{\sum_n |x[n]|^2}{\sum_n |x[n] - \hat{x}[n]|^2}\right) \text{ [dB]}
\end{equation}

\vspace{4pt}
\begin{table}[H]
\centering
\caption{Rezultate CFAR-STFT (100 MC)}
\label{tab:results_synthetic}
\begin{tabular}{ccccc}
\toprule
\textbf{SNR} & \textbf{RQF\_mean} & \textbf{RQF\_std} & \textbf{P\_d [\%]} & \textbf{N} \\
\midrule
5 & 7.28 & 0.47 & 100 & 100 \\
10 & 16.81 & 0.60 & 100 & 100 \\
15 & 22.95 & 0.56 & 100 & 100 \\
20 & 26.40 & 0.51 & 100 & 100 \\
25 & 28.43 & 0.39 & 100 & 100 \\
30 & 29.17 & 0.25 & 100 & 100 \\
\bottomrule
\end{tabular}
\vspace{-4pt}
\end{table}

\section{Analiză comparativă: Algoritmi CFAR și Clustering}

\subsection{Principii fundamentale CFAR}

Deși arhitecturile specifice variază, toți algoritmii CFAR (Constant False Alarm Rate) urmează principii comune: procesare prin fereastră glisantă (CUT + \textit{guard cells} + \textit{training cells}, iar guard este exclus din estimare pentru a evita „scurgerea” energiei unui target), prag adaptiv $T=\alpha\cdot Z$ (cu $\alpha$ dependent de $P_{fa}$ și $N_T$) și decizie de detecție dacă $P_{\text{CUT}}\ge T$.

\subsection{Algoritmi CFAR}

CA-CFAR: folosește media aritmetică, robust în zgomot (aproape) omogen. OS-CFAR: folosește o percentilă (în locul mediei), robust la interferențe/outlieri. SOCA-CFAR: alege minimul dintre mediile subregiunilor ferestrei; util la marginile de clutter.

\subsection{CFAR: Formule și pseudocod}

\begin{table}[H]
\centering
\small
\renewcommand{\arraystretch}{2.0}
\begin{tabular}{@{} l p{4cm} p{7.5cm} @{}}
\toprule
\textbf{Metodă} & \textbf{Model matematic} & \textbf{Pseudocod simplificat} \\ \midrule

CA-CFAR & 
$\begin{aligned}
Z &= \frac{1}{N_T} \sum_{i=1}^{N_T} x_i \\
T &= R \cdot Z
\end{aligned}$ & 
\textbf{1.} Extrage celulele de antrenament $S_{train}$ (fără zona de guard) \newline
\textbf{2.} $Z = \text{mean}(S_{train})$ \newline
\textbf{3.} IF $CUT \ge T$ THEN detecție \\ \midrule

OS-CFAR & 
$Z = x_{(k)}$, $T = \alpha \cdot Z$ &
\textbf{1.} Sortează $S_{train}$ → $S_{sorted}$ \newline
\textbf{2.} $Z = S_{sorted}[k]$ (valoarea de rang k) \newline
\textbf{3.} IF $CUT \ge T$ THEN detecție \\ \midrule

SOCA-CFAR &
$\begin{aligned}
Z &= \min(\mu_1, \dots, \mu_4) \\
T &= R \cdot Z
\end{aligned}$ &
\textbf{1.} Împarte fereastra în 4 subregiuni \newline
\textbf{2.} Calculează media $\mu_i$ pentru fiecare subregiune \newline
\textbf{3.} $Z = \min(\mu_{1}, \dots, \mu_4)$ \newline
\textbf{4.} IF $CUT \ge T$ THEN detecție \\ 

\bottomrule
\end{tabular}
\caption{Formule și pseudocod pentru metodele CFAR}
\end{table}

\subsection{Metode de clustering}

Observații: \textbf{Agglomerative} nu elimină zgomotul și poate grupa detecții false CFAR; \textbf{HDBSCAN} folosește Mutual Reachability Distance (MRD) pentru a penaliza zonele cu densitate mică, reducând fenomenul de \textit{chaining}.

\begin{table}[H]
\centering
\small
\renewcommand{\arraystretch}{1.6}
\begin{tabular}{@{} l p{3.5cm} p{7.5cm} @{}}
\toprule
\textbf{Metodă} & \textbf{Descriere conceptuală} & \textbf{Pseudocod} \\ \midrule

Agglomerative & 
Ierarhic \textit{bottom-up}. Fuzionează succesiv perechile cele mai apropiate. & 
\textbf{1.} Normalizează punctele $(f, t)$ \newline
\textbf{2.} Fiecare punct = cluster individual \newline
\textbf{3.} WHILE nr\_clustere $> K$: \newline
\hspace{1em} Găsește clusterele cu $d_{min}$ \newline
\hspace{1em} Le unește sub aceeași etichetă \newline
\textbf{4.} Reindexează etichetele (0, 1, 2...) \\ \midrule

	HDBSCAN & 
	Bazat pe densitate ierarhică. Folosește \textit{core distance} și \textit{MST} pentru stabilitate. & 
	\begin{tabular}[t]{@{}l@{}}
	\textbf{1.} Normalizează punctele $(f, t)$ \\
	\textbf{2.} Calculează \textbf{core distance} $(c)$ pentru fiecare punct \\
	\textbf{3.} Determină \textbf{MRD}: \\[0.35em]
	$d_{mrd}(u,v) = \max(c(u), c(v), d(u,v))$ \\
	\textbf{4.} Construiește \textbf{MST} folosind MRD \\
	\textbf{5.} Extrage componente conexe cu prag $\varepsilon$ \\
	\textbf{6.} Elimină clusterele sub $min\_samples$ $\to$ zgomot (-1)
	\end{tabular} \\ \midrule

	DBSCAN & 
	Bazat pe densitate & 
	\begin{tabular}[t]{@{}l@{}}
	Grupeaza puncte cu distanță $\le \varepsilon$. \\
	Asimetric: \\[0.25em]
	$d = \sqrt{\Delta t^2 + (\Delta f / \text{freq\_scale})^2}$
	\end{tabular} \\
	\bottomrule
	\end{tabular}
	\caption{Metode de clustering utilizate}
	\end{table}

\subsection{Separare prin triangulare Delaunay}

Alternativă geometrică la CFAR + clustering:

\begin{table}[H]
\centering
\scriptsize
\renewcommand{\arraystretch}{1.05}
\begin{tabular}{@{} p{2.1cm} p{4.4cm} p{6.6cm} @{}}
\toprule
\textbf{Pas} & \textbf{Idee} & \textbf{Pseudocod simplificat} \\ \midrule

Detectare \newline Vârfuri &
Maxime locale în spectrograma STFT peste un prag percentilă ($S=|X|^2$) &
\begin{tabular}[t]{@{}l@{}}
1. STFT $\rightarrow$ $S=|X|^2$ \\
2. Threshold $T$ = magnitudinea percentilei din $S$ \\
3. Maxime locale cu $S>T$ \\
4. Salvează coordonate vârfuri
\end{tabular} \\ \midrule

Triangulare \newline Delaunay &
Triangulare pe vârfuri cu criteriul cercului circumscris (Delaunay) &
\begin{tabular}[t]{@{}l@{}}
1. $T_1$: triunghi auxiliar (acoperă punctele) \\
2. Pentru fiecare $p$: inserează; re-triangulează local \\
3. La final: șterge triunghiurile cu vârfuri din $T_1$
\end{tabular} \\ \midrule

Gruparea \newline triunghiurilor &
Triunghiuri vecine conectate după energie medie, dacă au muchie comună & 
\begin{tabular}[t]{@{}l@{}}
1. Construiește lista de adiacență \\
2. Leagă triunghiuri cu energie similară ($\Delta<\varepsilon$) \\
3. Componente conexe (DFS/BFS) \\
4. Pentru fiecare componentă: centroid + energie
\end{tabular} \\
\bottomrule
\end{tabular}
\caption{Sinteză a pașilor de separare prin triangulare Delaunay și pseudocod asociat}
\end{table}

\subsection{Experimente pe semnale sintetice controlate}

\textbf{Metodă 1: CA-CFAR + HDBSCAN:} STFT Hamming, CA-CFAR, HDBSCAN clustering, reconstrucție componentă, evaluare RQF.

\textbf{Metodă 2: Triangulare Delaunay (separare geometrică):}
Se calculează STFT cu fereastră Gaussiană și se formează spectrograma $S=|X|^2$ în plan timp--frecvență. Apoi, în locul unei pragări CFAR, se extrag maxime locale peste un prag percentilă, obținând un set rar de vârfuri candidate. Pe coordonatele acestor vârfuri se construiește triangularea Delaunay, iar triunghiurile rezultate sunt grupate în componente conexe (pe baza adiacenței) și filtrate după energie medie similară. În final, Doppler-ul este estimat din componenta cu energia maximă (de ex. prin centroid/medie pe axa frecvență, rezultând $f_D$).

\textbf{Rezultate sintetice:}
CA-CFAR+HDBSCAN: 100\% detecție, RQF stabil SNR 5-30 dB. Triangulare: bună SNR mediu/ridicat, variabilitate mai mare RQF.

\begin{figure}[H]
\centering
\vspace{-4pt}
\includegraphics[width=0.495\textwidth]{../extensions/extensions/results/paper_extension_ca_hdbscan_hamming.pdf}%
\includegraphics[width=0.495\textwidth]{../extensions/extensions/results/paper_extension_triangulation.pdf}
\vspace{-6pt}
\caption{Extensie pe semnalul sintetic din lucrarea de referință: RQF vs. SNR pentru (stânga) CA-CFAR + HDBSCAN (fereastră Hamming) și (dreapta) separare geometrică prin triangulare Delaunay.}
\label{fig:paper_extension_rqf}
\vspace{-6pt}
\end{figure}

\section{Experimente pe IPIX cu target-uri reale}

S-au efectuat experimente pe date reale din baza IPIX. Figurile de mai jos arată cadre reprezentative din secvențele de detecție:

\begin{figure}[H]
\centering
\vspace{-4pt}
\includegraphics[width=0.85\textwidth]{../results/ipix_figures/ipix_target_17_goca_frame83.pdf}
\caption{GOCA-CFAR pe IPIX Target \#17: spectrograma (stânga), heatmap detecții (centru), detecții cadru (dreapta). Linie verticală = target la Doppler pozitiv.}
\label{fig:goca_detection}
\vspace{-4pt}
\end{figure}

\begin{figure}[H]
\centering
\vspace{-4pt}
\includegraphics[width=0.85\textwidth]{../results/ipix_figures/ipix_target_17_fractal_frame83.pdf}
\caption{GOCA-CFAR cu Fractal Boost pe IPIX Target \#17: Hurst exponent boost îmbunătățește detecția target-urilor slabe.}
\label{fig:fractal_boost_detection}
\vspace{-4pt}
\end{figure}

\subsection{Setul de Date și Scenarii}

Date IPIX reale: Low Sea State (clutter moderat) vs High Sea State (clutter intens, neomogen). Procesare segmentată 1s, PRF=1000 Hz, I/Q complex.

\subsection{Rezultate comparative: CA-CFAR+HDBSCAN vs. Triangulare Delaunay}

\begin{table}[H]
\centering
\small
\begin{tabular}{@{} l c c c @{}}
\toprule
\textbf{Metodă} & \textbf{Sea State} & \textbf{Componente} & \textbf{Viteza [m/s]} \\ \midrule
\multirow{2}{*}{CA-CFAR + HDBSCAN} 
    & HIGH & $1.0 \pm 0.0$ & $-0.054$ \\
    & LOW  & $1.0 \pm 0.0$ & $-0.008$ \\ \midrule
\multirow{2}{*}{Triangulare Delaunay} 
    & HIGH & $4.2$ & $+1.30$ \\
    & LOW  & $13.6$ & $-0.30$ \\ 
\bottomrule
\end{tabular}
\caption{Performanță comparativă pe IPIX (30 segmente × 1s per scenariu)}
\end{table}

\subsection{Observații Experimentale Detaliate}

\textbf{CA-CFAR + HDBSCAN:} \textbf{Problemă:} Produce detecții dominate de clutter; target-ul nu este izolat ($\sigma = 0$). HDBSCAN grupează tot clutter-ul într-un cluster unic. \textbf{Concluzie:} separă greu target-ul de clutter neomogen.

\textbf{Triangulare Delaunay:} Detecție ridicată dar instabilă. HIGH sea: 4.2 comp., LOW sea: 13.6 comp. (neașteptat). Doppler variază 4×: +1.30 vs. -0.30 m/s. Metoda geometrică fragmentează clutter-ul în multe componente, reducând consistența detecțiilor. \textbf{Concluzie:} Variabilitate prea mare.

\subsection{Concluzie validată experimental}

Pentru date reale neomogene, \textbf{CA-CFAR + HDBSCAN} produce detecții dominate de clutter (target-ul nu este izolat), \textbf{Triangulare Delaunay} oferă detecție geometrică dar instabilă (fragmentare excesivă), iar \textbf{GOCA-CFAR + DBSCAN anisotropic} (această implementare) oferă performanță optimă cu detecții stabile.

\textbf{Motivul reușitei:} (1) GOCA adaptează pragul local pe 4 cadrane de training (colțuri), (2) DBSCAN anisotropic ($s_f=3.0$, $s_t=1.5$) păstrează coeziunea target-urilor (linii verticale), (3) K-distribution + extensia Hurst reduc alarmele false generate de clutter.


\label{chap:adaptations}

Modificări cheie pentru adaptarea algoritmului din articol pe date reale sunt sintetizate în tabelul de mai jos.

\begin{table}[H]
\centering
\scriptsize
\renewcommand{\arraystretch}{1.05}
\begin{tabular}{@{} l p{5.2cm} p{6.2cm} @{}}
\toprule
\textbf{Metodă} & \textbf{Model / idee} & \textbf{Pași de implementare} \\ \midrule

K-distribution (clutter) &
Datele marine au \textit{(heavy tails)}: în „margini” apar valori cu ponderi mari; puterea $Z=|X(k,n)|^2$ este mai bine modelată de o distribuție de tip K decât de una Gaussiană/exponențială &
Parametrii K se estimează o singură dată din celule de training (cache); pragul se obține din ICDF pentru $P_{fa}$ și se aplică adaptiv local prin GOCA ($\hat{Z}=\max(\mu_i)$). \\ \midrule

Hartă Hurst (fractalitate) &
Separă target-uri slabe ($H<0.6$) de clutter ($H\approx 0.8$) &
Extensie opțională: (\textit{time}) $H$ pe ferestre în timp (pe $|x[n]|$) și proiectare pe axa timp STFT, sau (\textit{tf}) $H$ pe patch-uri TF. Masca $M_H$ se combină cu detecțiile CFAR. \\ \midrule

DBSCAN anisotropic + măști &
Păstrează semnăturile verticale și taie DC &
Distanță în bin-uri normalizate: $d = \sqrt{(s_t\Delta t)^2 + (\Delta f/s_f)^2}$, cu $s_f=3$, $s_t=1.5$. (Opțional, în animația IPIX: mascare DC și filtru pe lățime Doppler.) \\

\bottomrule
\end{tabular}
\caption{Adaptări ale algoritmului CFAR-STFT pentru date reale: model statistic, fractalitate și filtrare de cluster.}
\end{table}

% ============================================================================
% CAPITOLUL 4: DETALII DE IMPLEMENTARE
% ============================================================================
\chapter{Detalii de implementare}

% Revenim la spatierea globala.
\titlespacing{\section}{0pt}{2pt}{1pt}
\titlespacing{\subsection}{0pt}{1pt}{0pt}

\section{Framework și tehnologii}

Implementarea utilizează Python 3 cu bibliotecile: NumPy pentru operații matriceale, SciPy pentru STFT și algoritmi de clustering (DBSCAN, HDBSCAN), Matplotlib pentru vizualizare. Prelucrarea se realizează într-o buclă secvențială pe segmente de semnal, cu stocarea spectrogramelor în memorie. Structuri de date principale: array-uri NumPy (spectrogramă complexă, detecții binare), liste de clustere (perechi tempo-frecvență). Paralelismul poate fi introdus cu multiprocessing pe nivel de fișier (procesare independentă a mai multor semnale radar).

\section{Parametri și calibrare}

\begin{table}[H]
\centering
\caption{Parametrii algoritmului -- valori utilizate}
\begingroup
\renewcommand{\arraystretch}{1.15}
\setlength{\tabcolsep}{6pt}
\begin{tabular}{lccc}
\toprule
\textbf{Parametru} & \textbf{Valoare} & \textbf{Interval} & \textbf{Semnificație} \\
\midrule
$N_{fft}$ (window\_size) & 256 & $[128, 512]$ & Lungime FFT \\
$H$ (hop\_size) & 32 & $[N/8, N/2]$ & Hop = 87.5\% overlap \\
$\sigma_{window}$ & 8 & [4, 16] & Deviația std. a ferestrei \\
$P_{fa}$ & $10^{-3}$ & [$10^{-4}$, $10^{-2}$] & Probabilitate alarmă falsă \\
$N_G$ (cfar\_guard) & 3 & [2, 8] & Dimensiune celule guard \\
$N_T$ (cfar\_training) & 12 & [8, 24] & Dimensiune celule training \\
$\varepsilon_{DBSCAN}$ & 8 & [4, 16] & Raza de clustering \\
minSamples & 5 & [3, 10] & Puncte minime per cluster \\
freq\_scale & 3.0 & [2, 5] & Scalare metrică anisotropică DBSCAN \\
\bottomrule
\end{tabular}
\endgroup
\end{table}


\section{Notă asupra Doppler}

$v_r = \frac{f_d \cdot c}{2 f_{RF}}$. IPIX: $f_d = +100$ Hz → $v_r \approx +1.6$ m/s, max $\pm 8$ m/s.

% ============================================================================
% CAPITOLUL 5: CONCLUZII ȘI DIRECȚII VIITOARE
% ============================================================================
\chapter{Concluzii și direcții viitoare}

\section{Concluzii}

Am obținut RQF = 29.17 dB (SNR = 30 dB, $P_d = 100\%$), cu performanță stabilă în intervalul 5--30 dB, validare pe date IPIX și replicare independentă prin repository-ul GitHub.

\section{Direcții viitoare}

\small
Direcții viitoare includ accelerare pe GPU, calibrare automată a parametrilor, urmărire multi-target și extinderea spre arhitecturi hibride (modelare fizică + \textit{data-driven}):
\begin{enumerate}[leftmargin=*,topsep=2pt,itemsep=1pt,parsep=0pt]
\item \textbf{Clasificare pe trăsături (Random Forest / XGBoost)}: după detecția CFAR, fiecare cluster poate fi descris printr-un vector de trăsături (energie, arie/durată, centroid $(t,f)$, Hurst, parametrii K, lățime Doppler), pentru separare non-liniară clutter/target.
\item \textbf{Modele secvențiale (LSTM/GRU)}: exploatează continuitatea cinematică; detecția se poate reformula ca \textit{predictive anomaly detection} pe serii STFT (eroare de predicție mare $\Rightarrow$ anomalie).
\item \textbf{Autoencodere convoluționale (CAE)}: antrenare pe clutter pur; reziduul de reconstrucție produce o hartă de eroare care poate ghida (atenționa) pragarea CFAR în zonele suspecte.
\item \textbf{Tracking multi-target}: extindere \textit{track-while-scan} (ex. filtre de particule / PHD) pentru menținerea identității și predicție în intervale cu fading.
\item \textbf{Accelerare hardware}: portare STFT/CFAR 2D pe GPU (CUDA) sau FPGA pentru procesare aproape în timp real pe fluxuri I/Q.
\end{enumerate}
\normalsize

\section{Cod}

Repository GitHub: \url{https://github.com/dirgnic/Radar_Detection_STFT} pentru replicare independentă.

\begin{thebibliography}{99}

\bibitem{abratkiewicz2022}
Abratkiewicz, K. (2022).
\textit{Radar Detection-Inspired Signal Retrieval from the Short-Time Fourier Transform}.
Sensors, 22(16), 5954.
\newline \url{https://doi.org/10.3390/s22165954}

\bibitem{ipix}
S. Haykin, et al.,
``IPIX Radar Database,''
McMaster University / DREO, 1993.
\url{http://soma.ece.mcmaster.ca/ipix/}

\bibitem{ward2006}
K. D. Ward, R. J. A. Tough, S. Watts,
\textit{Sea Clutter: Scattering, the K Distribution and Radar Performance},
IET, 2006.

\bibitem{hurst1951}
H. E. Hurst,
``Long-term storage capacity of reservoirs,''
\textit{Trans. Am. Soc. Civil Eng.}, vol. 116, pp. 770-799, 1951.

\bibitem{harris1978}
Harris, F.J. (1978).
\textit{On the Use of Windows for Harmonic Analysis with the Discrete Fourier Transform}.
Proceedings of the IEEE, 66(1), 51-83.

\bibitem{ester1996}
Ester, M., Kriegel, H.-P., Sander, J., Xu, X. (1996).
\textit{A Density-Based Algorithm for Discovering Clusters in Large Spatial Databases with Noise}.
In KDD'96: Proceedings, pp. 226-231.

\bibitem{rohling1983}
H. Rohling,
``Radar CFAR thresholding in clutter and multiple target situations,''
\textit{IEEE Trans. Aerospace Electron. Syst.}, vol. 19, no. 4, 1983.

\bibitem{richards2005}
Richards, M. A. (2005).
\textit{Fundamentals of Radar Signal Processing}.
McGraw-Hill Professional.

\bibitem{triangulation2023}
Triangulation Separation Method,
\textit{IPIX Radar Target Separation via Delaunay Triangulation}.
Available: Included in \texttt{extensions/triangulation\_separation.py}

\end{thebibliography}

\end{document}
