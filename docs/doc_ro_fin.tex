% Documentație conform template FMI (report, Times, 1.5 line spacing)
% Autori: Ingrid Corobana, Teodora Nae
% Compilare recomandată: xelatex doc_ro_fin.tex

\documentclass[12pt, a4paper]{report}

% Suport pentru diacritice și alte simboluri
\usepackage{fontspec}

% Font Times New Roman
\usepackage{times}

% Suport pentru mai multe limbi
\usepackage{polyglossia}

% Setează limba textului la română
\setdefaultlanguage{romanian}
% Am nevoie de engleză pentru rezumat
\setotherlanguages{english}

% Indentează și primul paragraf al fiecărei noi secțiuni
\SetLanguageKeys{romanian}{indentfirst=true}

% Suport pentru diferite stiluri de ghilimele
\usepackage{csquotes}

\DeclareQuoteStyle{romanian}
  {\quotedblbase}
  {\textquotedblright}
  {\guillemotleft}
  {\guillemotright}

% Setează spațiere inter-linie (doublespacing din LaTeX este echivalentul setarii 1.5 in Microsoft Word)
\usepackage{setspace}
\doublespacing

% Modificarea geometriei paginii (cu margini de 2,5 cm) 
\usepackage[margin=2.5cm]{geometry}

% Include funcțiile de grafică
\usepackage{graphicx}
% Încarcă imaginile din directorul `images`
\graphicspath{{./images/}{../results/ipix_figures/}}

% Pachete matematice
\usepackage{amsmath}
\usepackage{amssymb}

% Algoritmi
\usepackage{algorithm}
\usepackage{algpseudocode}

% Listări de cod
\usepackage{listings}

% Culori
\usepackage{xcolor}

% Linkuri interactive în PDF
\usepackage[
    colorlinks,
    linkcolor={black},
    menucolor={black},
    citecolor={black},
    urlcolor={blue}
]{hyperref}

% Tabele
\usepackage{array}
\usepackage{booktabs}
\usepackage{float}
\usepackage{caption}

% Pentru includerea PDF-urilor
\usepackage{pdfpages}

% Pachete pentru figurile complexe TikZ
\usepackage{tikz}
\usetikzlibrary{arrows.meta,positioning,fit,calc,shadows.blur,backgrounds}

% Stiluri diferite de headere și footere
\usepackage{fancyhdr}

\definecolor{stftblue}{RGB}{66, 133, 244}
\definecolor{cfarorange}{RGB}{255, 152, 0}
\definecolor{dbscangreen}{RGB}{76, 175, 80}
\definecolor{darkblue}{RGB}{25, 50, 100}
\definecolor{codegray}{RGB}{240, 240, 240}

\lstset{
    language=Python,
    basicstyle=\ttfamily\small,
    backgroundcolor=\color{codegray},
    breaklines=true,
    keywordstyle=\color{blue},
    commentstyle=\color{gray},
    stringstyle=\color{red},
    identifierstyle=\color{darkblue},
    numbers=left,
    numberstyle=\tiny\color{gray}
}

\renewcommand{\algorithmicrequire}{\textbf{Input:}}
\renewcommand{\algorithmicensure}{\textbf{Output:}}

% Metadate
\title{Analiza semnalelor radar în prezența ecourilor marine (sea clutter)}
\author{Ingrid Corobana \quad Teodora Nae}

% Generează variabilele cu @
\makeatletter

% Suport pentru rezumat în două limbi
\newenvironment{abstractpage}
  {\cleardoublepage\vspace*{\fill}\thispagestyle{empty}}
  {\vfill\cleardoublepage}
\renewenvironment{abstract}[1]
  {\bigskip\selectlanguage{#1}%
   \begin{center}\bfseries\abstractname\end{center}}
  {\par\bigskip}

\begin{document}

% ============================================================================
% PAGINA DE TITLU - Format FMI
% ============================================================================
\begin{titlepage}
\newgeometry{left=2cm,right=2cm,top=1.5cm,bottom=1cm}
\begin{singlespace}

\begin{figure}[!htb]
    \centering
    \begin{minipage}{0.18\textwidth}
        \includegraphics[width=\linewidth]{logo-ub.png}
    \end{minipage}
    \begin{minipage}{0.55\textwidth}
        \begin{center}
            \textbf{UNIVERSITATEA DIN BUCUREȘTI}\\[0.2cm]
            \textbf{FACULTATEA DE MATEMATICĂ ȘI INFORMATICĂ}
        \end{center}
    \end{minipage}
    \begin{minipage}{0.18\textwidth}
        \includegraphics[width=\linewidth]{logo-fmi.png}
    \end{minipage}
\end{figure}

\vspace{0.3cm}
\begin{center}
\textbf{SPECIALIZAREA INFORMATICĂ}
\end{center}

\vspace{0.8cm}

\begin{center}
\Large \textbf{Proiect Procesarea Semnalelor}
\end{center}

\vspace{0.5cm}

\begin{center}
\LARGE \textbf{ANALIZA SEMNALELOR RADAR ÎN PREZENȚA ECOURILOR MARINE}
\end{center}

\vspace{0.3cm}

\begin{center}
\large Abordare bazată pe CFAR-STFT și experimente pe date sintetice și reale
\end{center}

\vspace{1.5cm}

\begin{center}
\large \textbf{Studenți}\\[0.1cm]
Ingrid Corobana\\
Teodora Nae
\end{center}

\vspace{0.8cm}

\begin{center}
\large \textbf{Coordonator științific}\\[0.1cm]
Conf. Dr. Cristian Rusu
\end{center}

\vspace{1.2cm}

\begin{center}
\small \textbf{Repository GitHub:} \url{https://github.com/dirgnic/Radar_Detection_STFT}
\end{center}

\vspace{0.8cm}

\begin{center}
\Large \textbf{București, 2026}
\end{center}

\end{singlespace}
\end{titlepage}

\restoregeometry
\newgeometry{margin=2.5cm}

% ============================================================================
% PAGINA DE REZUMAT
% ============================================================================
\begin{abstractpage}

\begin{abstract}{romanian}
Acest document prezintă o implementare completă a algoritmului \textbf{CFAR--STFT}, propus de \textbf{Abratkiewicz (2022)}, pentru detecția și recuperarea semnalelor radar în prezența zgomotului și a clutter-ului maritim. Algoritmul combină \textbf{Short-Time Fourier Transform (STFT)}, \textbf{detecție adaptivă CFAR 2D}, \textbf{clustering DBSCAN} și \textbf{dilatare geodezică} pentru a reconstrui semnale radar cu fidelitate ridicată.

Implementarea este validată pe date sintetice (chirp neliniar) și pe date reale (\textbf{IPIX radar sea clutter}). Pe semnalul sintetic controlat, algoritmul detectează componenta de interes în toate cele 100 de rulări Monte Carlo (rata de detecție 100\%). \textbf{RQF} (Reconstruction Quality Factor) variază de la \textbf{7.28 dB} la \textbf{SNR=5 dB} până la \textbf{29.17 dB} la \textbf{SNR=30 dB}.

\textbf{Contribuția cheie}: adaptăm algoritmul la sea clutter real folosind \textbf{K-distribution} (în loc de Gaussian), \textbf{fractal/Hurst boost} pentru ținte slabe și \textbf{DBSCAN asimetric} pentru clustering de semnături verticale.

Cod sursă disponibil la: \url{https://github.com/dirgnic/Radar_Detection_STFT}

\textbf{Cuvinte cheie:} CFAR, STFT, radar, sea clutter, K-distribution, DBSCAN, detecție adaptivă
\end{abstract}

\newpage

\begin{abstract}{english}
This document presents a complete implementation of the \textbf{CFAR-STFT} algorithm, proposed by \textbf{Abratkiewicz (2022)}, for detection and recovery of radar signals in the presence of noise and maritime clutter. The algorithm combines \textbf{Short-Time Fourier Transform (STFT)}, \textbf{2D adaptive CFAR detection}, \textbf{DBSCAN clustering}, and \textbf{geodesic dilation} to reconstruct radar signals with high fidelity.

The implementation is validated on synthetic data (nonlinear chirp) and real data (\textbf{IPIX radar sea clutter}). On controlled synthetic signal, the algorithm detects the component of interest in all 100 Monte Carlo runs (100\% detection rate). \textbf{RQF} (Reconstruction Quality Factor) ranges from \textbf{7.28 dB} at \textbf{SNR=5 dB} to \textbf{29.17 dB} at \textbf{SNR=30 dB}.

\textbf{Key contribution}: we adapt the algorithm to real sea clutter using \textbf{K-distribution} (instead of Gaussian), \textbf{fractal/Hurst boost} for weak targets, and \textbf{asymmetric DBSCAN} for vertical signature clustering.

\textbf{Keywords:} CFAR, STFT, radar, sea clutter, K-distribution, DBSCAN, adaptive detection
\end{abstract}

\end{abstractpage}

% ============================================================================
% CUPRINS
% ============================================================================
\tableofcontents

% ============================================================================
% CONFIGURARE PAGINI PRINCIPALE
% ============================================================================
\cleardoublepage
\fancypagestyle{main}{
  \fancyhf{}
  \renewcommand\headrulewidth{0pt}
  \fancyhead[C]{}
  \fancyfoot[C]{\thepage}
}
\pagestyle{main}

% ============================================================================
% CAPITOLUL 1: INTRODUCERE
% ============================================================================
\chapter{Introducere}

Problema principală pe care o rezolvăm este \textbf{detecția obiectelor mici în date radar maritime}, într-un mediu complex care se schimbă constant din cauza valurilor. Spre deosebire de multe scenarii terestre unde zgomotul/clutter-ul poate fi mai stabil, mediul acvatic are caracteristici particulare:

\begin{itemize}
    \item statisticile \textbf{nu sunt bine modelate Gaussian} (avem distribuții cu cozi grele și amplitudini mari),
    \item există \textbf{corelație temporală} (valurile creează tipare structurate),
    \item efectele Doppler duc la \textbf{extinderea spectrului} (valuri în mișcare),
    \item apar \textbf{spike-uri} vizibile în spectrograme când valurile sunt mai mari.
\end{itemize}

Metodele tradiționale adaptive de detecție \textbf{CFAR} (Constant False Alarm Rate) pot fi limitate atunci când pierd informația temporală și nu exploatează structura timp--frecvență. Abratkiewicz (2022)\footnote{Abratkiewicz, K. (2022). Radar Detection-Inspired Signal Retrieval from the Short-Time Fourier Transform. Sensors, 22(16), 5954.} propune o abordare care folosește explicit structura \textbf{time--frequency} pentru a îmbunătăți atât \textbf{detecția}, cât și \textbf{recuperarea/reconstrucția} componentelor semnalului.

\section{Obiectivele proiectului}

În acest proiect ne propunem:

\begin{enumerate}
    \item implementarea completă a algoritmului \textbf{CFAR--STFT} în \textbf{Python};
    \item validarea pe date sintetice: \textbf{chirp neliniar} conform Ecuației (14);
    \item testarea pe date reale: \textbf{IPIX} cu sea clutter complex;
    \item analiză Doppler pentru estimarea vitezei (și interpretarea țintelor);
    \item documentație detaliată și reproductibilitate;
    \item validare rezultate: detecție și reconstrucție;
    \item \textbf{adaptări sea clutter}: K-distribution, boost fractal (Hurst), DBSCAN asimetric.
\end{enumerate}

\section{Structura documentului}

Documentul este organizat astfel:

\begin{itemize}
    \item \textbf{Secțiunea 2}: fundamente teoretice și formule esențiale;
    \item \textbf{Secțiunea 3}: pașii algoritmului + pseudocod;
    \item \textbf{Secțiunea 4}: surse de date și validare (IPIX);
    \item \textbf{Secțiunea 5}: rezultate experimentale (sintetic + real);
    \item \textbf{Secțiunea 6}: adaptări pentru sea clutter;
    \item \textbf{Secțiunea 7}: detalii implementare;
    \item \textbf{Secțiunea 8}: concluzii și direcții viitoare.
\end{itemize}

% ============================================================================
% DIAGRAME PIPELINE - incluse după Introducere
% ============================================================================
\section{Diagrame Pipeline CFAR-STFT}

Următoarele pagini prezintă diagramele vizuale ale pipeline-ului algoritmului CFAR-STFT, incluzând structura detectorului GOCA-CFAR 2D, configurația experimentală și comparația parametrilor.

\includepdf[pages=1,scale=0.95,offset=0 -20,pagecommand={}]{pipeline_diagrams.pdf}
\includepdf[pages=2-,scale=0.95,offset=0 -20,pagecommand={}]{pipeline_diagrams.pdf}

% ============================================================================
% CAPITOLUL 2: FUNDAMENTE TEORETICE
% ============================================================================
\chapter{Fundamente teoretice}

\section{Short-Time Fourier Transform (STFT)}

STFT este baza algoritmului și produce o reprezentare \textbf{timp--frecvență} a semnalului:

\begin{equation}
X(k,n) = \sum_{m=0}^{N-1} x(m) \cdot w(m - nH) \cdot e^{-j2\pi km/N}
\end{equation}

unde:
\begin{itemize}
    \item $x(m)$ --- semnalul de intrare
    \item $w(\cdot)$ --- fereastra (Gaussiană, $\sigma=8$ bin-uri)
    \item $n$ --- indexul de timp (ferestre succesive)
    \item $k$ --- indexul de frecvență
    \item $N$ --- lungimea FFT (512)
    \item $H$ --- hop-size între ferestre (256, 50\% overlap)
\end{itemize}

\textbf{Fereastra Gaussiană}:
\begin{equation}
w(m) = e^{-m^2/(2\sigma^2)} \quad \text{cu} \quad \sigma=8
\end{equation}

Se alege pentru minimizarea spectral leakage.

\section{Detecție adaptivă CFAR 2D (GOCA-CFAR)}

CFAR adaptează pragul local pe baza nivelului de zgomot din vecinătate:

\begin{equation}
H(k,n) = \begin{cases}
1 & \text{dacă } |X(k,n)|^2 > \lambda \cdot \mathcal{N}(k,n) \\
0 & \text{altfel}
\end{cases}
\end{equation}

unde $\mathcal{N}(k,n)$ este estimarea zgomotului local.

În \textbf{GOCA-CFAR} (Greatest-Of Cell Averaging), zona de training se împarte în 4 cadrane și se ia maximul mediilor:

\begin{equation}
\hat{Z} = \max(\mu_1, \mu_2, \mu_3, \mu_4), \qquad T = R \cdot \hat{Z}
\end{equation}

\begin{equation}
R = N_T \left( P_f^{-1/N_T} - 1 \right)
\end{equation}

Decizie (în plan TF, pe puterea spectrogramelor):
\begin{equation}
|X(k,n)|^2 \ge T \Rightarrow \text{Detectat}, \quad |X(k,n)|^2 < T \Rightarrow \text{Zgomot}
\end{equation}

\section{DBSCAN pentru clustering}

După CFAR, punctele detectate sunt grupate cu \textbf{DBSCAN} (clustering pe densitate). Pentru semnături verticale tipice țintelor, folosim o \textbf{distanță asimetrică} cu toleranță mai mare pe frecvență (factor 3):

\begin{equation}
d = \sqrt{\Delta t^2 + \left(\frac{\Delta f}{3}\right)^2}
\end{equation}

% ============================================================================
% CAPITOLUL 3: DESCRIEREA ALGORITMULUI
% ============================================================================
\chapter{Descrierea completă a algoritmului}

\section{Pipeline general (5 pași)}

Algoritmul complet are cinci pași:

\begin{enumerate}
    \item calcul STFT cu fereastră Gaussiană;
    \item detecție \textbf{CFAR 2D} în plan timp--frecvență;
    \item clustering DBSCAN al punctelor detectate;
    \item extinderea măștii prin \textbf{dilatare geodezică} (geodesic dilation);
    \item reconstrucție prin \textbf{iSTFT} folosind masca (aplicată element-wise pe STFT).
\end{enumerate}

\subsection{Pasul 1: Calcul STFT}

\begin{algorithm}[H]
\caption{Calcul STFT cu fereastră Gaussiană}
\begin{algorithmic}[1]
\Require Semnal de intrare $x[n]$, lungime FFT $N_{fft}=512$, hop $H=256$, $\sigma=8$
\Ensure Matrice STFT $X_{stft} \in \mathbb{C}^{N_f \times N_t}$
\State $N_t \gets \lceil (len(x) - N_{fft}) / H \rceil + 1$
\State $X_{stft} \gets \text{zeros}(N_{fft}/2 + 1, N_t)$ \Comment{One-sided}
\State Precalculează fereastra Gaussiană: $w[m] \gets e^{-m^2/(2\sigma^2)}$
\For{$n \gets 0$ to $N_t - 1$}
    \State Extrage fereastră: $x_n \gets x[nH : nH + N_{fft}]$
    \State Aplică fereastră: $x_w \gets x_n \odot w$
    \State Calculează FFT: $X_n \gets \text{fft}(x_w, N_{fft})$
    \State Stochează one-sided: $X_{stft}[:, n] \gets X_n[0 : N_{fft}/2 + 1]$
\EndFor
\State Normalizează: $X_{stft} \gets X_{stft} / \sum_m w[m]^2$
\State \Return $X_{stft}$
\end{algorithmic}
\end{algorithm}

\subsection{Pasul 2: Detecție CFAR 2D}

\begin{algorithm}[H]
\caption{Detecție CFAR 2D (GOCA)}
\begin{algorithmic}[1]
\Require Matrice STFT $X_{stft}$, $P_f=0.001$, $N_G=3$, $N_T=12$
\Ensure Mască binară de detecție $H \in \{0,1\}^{N_f \times N_t}$
\State $H \gets \text{zeros}(N_f, N_t)$
\State $N_f \gets $ rows($X_{stft}$), $N_t \gets $ cols($X_{stft}$)
\For{$k \gets N_G + N_T$ to $N_f - N_G - N_T - 1$}
    \For{$n \gets N_G + N_T$ to $N_t - N_G - N_T - 1$}
        \State Extrage Training Cells în 4 cadrane în jurul $(k,n)$
        \State $\mu_i \gets \text{mean}(\text{cadran}_i)$ pentru $i \in \{1,2,3,4\}$
        \State $\mathcal{N}_{local} \gets \max(\mu_1, \mu_2, \mu_3, \mu_4)$ \Comment{GOCA}
        \State $\lambda \gets R(P_f) \cdot \mathcal{N}_{local}$
        \If{$|X_{stft}(k,n)|^2 > \lambda$}
            \State $H(k,n) \gets 1$
        \EndIf
    \EndFor
\EndFor
\State \Return $H$
\end{algorithmic}
\end{algorithm}

\subsection{Pasul 3: Clustering DBSCAN}

\begin{algorithm}[H]
\caption{Clustering DBSCAN în plan timp--frecvență}
\begin{algorithmic}[1]
\Require Puncte detectate $\{(f_i, t_i)\}_{i=1}^{N_p}$, $\varepsilon=8$, minSamples$=5$
\Ensure Etichete cluster $\text{labels} \in \mathbb{Z}$
\State $\text{labels} \gets -1 \cdot \text{ones}(N_p)$ \Comment{-1 = zgomot}
\State $C \gets 0$ \Comment{Index cluster curent}
\For{$i \gets 1$ to $N_p$}
    \If{labels$[i] \neq$ nevizitat}
        \State Continue
    \EndIf
    \State $N \gets \text{RangeQuery}(i, \varepsilon)$ \Comment{Vecini în rază}
    \If{$|N| < $ minSamples}
        \State labels$[i] \gets -1$ \Comment{Zgomot}
    \Else
        \State $C \gets C + 1$
        \State \textbf{ExpandCluster}$(i, C, N, \varepsilon, \text{minSamples})$
    \EndIf
\EndFor
\State \Return labels
\end{algorithmic}
\end{algorithm}

\subsection{Pasul 4: Dilatare geodezică}

\begin{algorithm}[H]
\caption{Dilatare geodezică pe mască}
\begin{algorithmic}[1]
\Require Mască binară $H$ (de la CFAR), iterații $n_{iter}=3$
\Ensure Mască dilatată $H_{dil}$
\State $H_{dil} \gets H$
\State Kernel $\gets$ binar $3 \times 3$ cruce
\For{$i \gets 1$ to $n_{iter}$}
    \State $H_{new} \gets \text{zeros}(H_{dil}.shape)$
    \For{$k \gets 1$ to rows($H_{dil}$) $-2$}
        \For{$n \gets 1$ to cols($H_{dil}$) $-2$}
            \State $H_{new}(k,n) = \max(H_{dil}(k-1,n), H_{dil}(k,n),$ 
            \State \hspace{3cm} $H_{dil}(k+1,n), H_{dil}(k,n-1), H_{dil}(k,n+1))$
        \EndFor
    \EndFor
    \State $H_{dil} \gets H_{new}$
\EndFor
\State \Return $H_{dil}$
\end{algorithmic}
\end{algorithm}

\subsection{Pasul 5: Reconstrucție iSTFT}

\begin{algorithm}[H]
\caption{Reconstrucție inversă (iSTFT)}
\begin{algorithmic}[1]
\Require STFT original $X_{stft}$, mască dilatată $H_{dil}$, fereastră $w$, hop $H=256$
\Ensure Semnal reconstruit $\hat{x}(n)$
\State $X_{masked} \gets X_{stft} \odot H_{dil}$ \Comment{Aplică masca element-wise}
\State $N_t \gets $ cols($X_{masked}$)
\State $M \gets N_{fft}$ \Comment{Lungimea semnalului reconstruit}
\State $\hat{x} \gets \text{zeros}(M)$
\For{$n \gets 0$ to $N_t - 1$}
    \State Calculează iFFT: $x_n \gets \text{ifft}(X_{masked}[:, n], N_{fft})$
    \State Aplică fereastră: $x_w \gets \text{real}(x_n) \odot w$
    \State Adună cu overlap-add: $\hat{x}[nH : nH + N_{fft}] \mathrel{+}= x_w$
\EndFor
\State Normalizează după fereastră: $\hat{x} \gets \hat{x} / (\sum_m w[m]^2)$
\State \Return $\hat{x}$
\end{algorithmic}
\end{algorithm}

% ============================================================================
% CAPITOLUL 4: DATE ȘI SURSE DE VALIDARE
% ============================================================================
\chapter{Date și surse de validare}

\section{Baza de date IPIX (radar maritim)}

Componenta fundamentală a proiectului este folosirea datelor reale din \textbf{IPIX} (Intelligent Pixel Processing for X-band radar), de la \textbf{McMaster University, Canada}. Datele sunt de la un radar coerent polarimetric X-band folosit pentru monitorizarea activității maritime.

\subsection{Caracteristici tehnice IPIX}

Caracteristici tehnice (esențiale pentru interpretarea Doppler și rezoluție):

\begin{itemize}
    \item \textbf{Frecvența RF}: $f_{RF} = 9.39$ GHz (X-band) -- optim pentru detectarea obiectelor mici
    \item \textbf{PRF (Pulse Repetition Frequency)}: 1000 Hz -- permite detecție Doppler
    \item \textbf{Lungime puls}: 200 ns -- rezoluție spațială mare
    \item \textbf{Lățime fascicul}: 0.9 grade -- precizie unghiulară excelentă
    \item \textbf{Format date}: Complex I/Q (In-phase + Quadrature)
\end{itemize}

\subsection{Ce sunt datele complexe I/Q?}

Fiecare eșantion radar este complex: $x(t) = I(t) + j \cdot Q(t)$.

\begin{itemize}
    \item \textbf{Componenta I (In-phase)}: Proiecția semnalului pe axa cosinus
    \item \textbf{Componenta Q (Quadrature)}: Proiecția pe axa sinus (decalată cu 90°)
    \item \textbf{Magnitudine}: $|x(t)| = \sqrt{I^2 + Q^2}$ -- intensitatea ecoului
    \item \textbf{Fază}: $\phi(t) = \arctan(Q/I)$ -- informația Doppler
\end{itemize}

Reprezentarea I/Q permite \textbf{detecția frecvențelor Doppler pozitive și negative}, esențială pentru:
\begin{itemize}
    \item Ținte care se apropie (frecvență Doppler pozitivă)
    \item Ținte care se îndepărtează (frecvență Doppler negativă)
    \item Clutter static (frecvență Doppler $\approx$ 0 Hz)
\end{itemize}

\subsection{Target real}

Ținta folosită în fișierele IPIX: o \textbf{sferă de $\sim$1m} învelită în plasă de sârmă, ancorată la \textbf{$\sim$2660m}.

\begin{table}[H]
\centering
\caption{Fișiere IPIX cu ținte reale}
\begin{tabular}{lccc}
\toprule
\textbf{Fișier} & \textbf{Range Cell} & \textbf{Polarizare} & \textbf{Stare mare} \\
\midrule
\#17 (19931106\_180557) & 7 & HH & Moderată \\
\#18 (19931106\_181048) & 7 & HH & Moderată \\
\#30 (19931106\_191449) & 7 & HH & Ridicată \\
\#40 (19931106\_195609) & 7 & HH & Moderată \\
\bottomrule
\end{tabular}
\end{table}

\subsection{Interpretarea spectrogramelor IPIX și „de ce arată ciudat"}

Spectrograma IPIX este \textbf{two-sided} (frecvențe negative/pozitive) pentru că datele sunt complexe I/Q. Elemente cheie:

\begin{itemize}
    \item \textbf{Linie roșie groasă la 0 Hz (DC)}: ecouri staționare de la suprafața mării
    \item \textbf{Verde/galben în jurul DC}: sea clutter activ (valuri/spumă/mișcare)
    \item \textbf{Zone albastre laterale}: zgomot termic slab uniform
    \item \textbf{Aspect „punctat"}: energia e în ridges (creste) TF, nu uniformă
\end{itemize}

\textbf{Ce este sea clutter?}

Sea clutter (zgomot maritim) reprezintă \textbf{ecouri radar de la suprafața mării}:

\begin{itemize}
    \item Reflexii de la valuri, spumă, picături de apă
    \item Concentrat în jurul frecvenței 0 Hz (Doppler mic -- mișcare lentă)
    \item Energie aproximativ 90--95\% din cazuri în intervalul $[-50, +50]$ Hz
    \item Are structură neuniformă -- unele zone sunt mai intense (ridges)
    \item Țintele reale (nave, obiecte) apar departe de DC ($\pm 100$--$400$ Hz)
\end{itemize}

\begin{figure}[H]
\centering
\includegraphics[width=0.95\textwidth]{../results/ipix_figures/ipix_seaclutter_explanation.png}
\caption{Explicație vizuală: ce sunt sea clutters în date IPIX? Figura arată spectrograma completă (two-sided), zoom pe regiunea clutter în jurul DC (frecvență 0 Hz) și distribuția energiei pe frecvențe. Notă: majoritatea energiei (>90\%) este concentrată în $\pm 50$ Hz în jurul DC.}
\label{fig:seaclutter_explain}
\end{figure}

\subsection{De ce CFAR nu detectează „toată zona albă"?}

Întrebare legitimă când te uiți la rezultate: \textit{„De ce detectează doar puncte împrăștiate și nu toată regiunea cu energie mare?"}. Răspunsul stă în natura \textbf{local-adaptivă} a CFAR:

\begin{itemize}
    \item \textbf{Prag global}: Un algoritm simplist ar seta un prag fix (ex: „detectează tot ce e peste -20 dB"). Asta ar detecta \textit{toată zona albă} --- dar ar detecta și zgomot puternic, generând mii de alarme false!
    
    \item \textbf{CFAR local-adaptiv}: Compară fiecare pixel cu \textit{vecinii săi} (training cells). Dacă pixelul e „mult peste media vecinilor locali" $\rightarrow$ detectat. Altfel $\rightarrow$ ignorat.
\end{itemize}

\textbf{Consecință}: Într-o zonă uniformă de energie mare (sea clutter), toți pixelii au \textit{vecini la fel de puternici} $\rightarrow$ CFAR nu îi detectează, pentru că nu sunt „outlieri" relativ la contextul local! CFAR detectează doar:
\begin{itemize}
    \item \textbf{Margini} unde energia crește brusc
    \item \textbf{Ridges} (creste de energie) care depășesc clutter-ul înconjurător
    \item \textbf{Ținte reale} care ies din zgomot/clutter
\end{itemize}

De aceea pare că ia „puncte aleatorii" --- în realitate, ia exact punctele care reprezintă \textit{tranziții de putere}!

Aceasta e \textit{puterea} CFAR: reduce dramatic rata de alarme false adaptându-se la zgomot. Prețul plătit: nu mai detectăm „zone întregi", doar puncte care conțin informație nouă (schimbări bruște în spectrogramă).

% ============================================================================
% CAPITOLUL 5: REZULTATE EXPERIMENTALE
% ============================================================================
\chapter{Rezultate experimentale}

\section{Experimente pe semnale sintetice}

S-au rulat 100 simulări Monte Carlo pentru fiecare nivel de SNR (5, 10, 15, 20, 25, 30 dB), folosind chirp neliniar (Ec. 14). Rata de detecție: 100\% în toate rulările.

Metrica RQF:
\begin{equation}
\text{RQF} = 10 \log_{10}\left(\frac{\sum_n |x[n]|^2}{\sum_n |x[n] - \hat{x}[n]|^2}\right) \text{ [dB]}
\end{equation}

\begin{table}[H]
\centering
\caption{Rezultate CFAR-STFT pe chirp neliniar sintetic -- 100 rulări MC}
\label{tab:results_synthetic}
\begin{tabular}{ccccc}
\toprule
\textbf{SNR [dB]} & \textbf{RQF\_mean [dB]} & \textbf{RQF\_std [dB]} & \textbf{P\_detecție [\%]} & \textbf{N\_rulări} \\
\midrule
5 & 7.28 & 0.47 & 100.0 & 100 \\
10 & 16.81 & 0.60 & 100.0 & 100 \\
15 & 22.95 & 0.56 & 100.0 & 100 \\
20 & 26.40 & 0.51 & 100.0 & 100 \\
25 & 28.43 & 0.39 & 100.0 & 100 \\
30 & 29.17 & 0.25 & 100.0 & 100 \\
\bottomrule
\end{tabular}
\end{table}

\section{Experimente pe IPIX cu ținte reale}

Am rulat detecție animată pe fișierele cu ținte reale. Figurile de mai jos arată cadre din animațiile de detecție:

\begin{figure}[H]
\centering
\includegraphics[width=0.95\textwidth]{../results/ipix_figures/ipix_target_17_goca_frame83.png}
\caption{Detecție GOCA-CFAR pe IPIX Target \#17 -- Cadru din animație arătând detecții active. Cele trei panouri arată: (stânga) spectrograma cu detecții acumulate (overlay roșu), (centru) heatmap de detecție, (dreapta) detecțiile cadrului curent. Linia verticală luminoasă reprezintă ținta plutitoare la Doppler pozitiv (se apropie de radar).}
\label{fig:goca_detection}
\end{figure}

\begin{figure}[H]
\centering
\includegraphics[width=0.95\textwidth]{../results/ipix_figures/ipix_target_17_fractal_frame83.png}
\caption{GOCA-CFAR cu Fractal Boost pe IPIX Target \#17 -- Cadru arătând detecții active. Fractal boost folosește analiza exponentului Hurst pentru a detecta ținte care perturbă structura self-similar a sea clutter, îmbunătățind detecția țintelor slabe.}
\label{fig:fractal_boost_detection}
\end{figure}

% ============================================================================
% CAPITOLUL 6: ADAPTĂRI PENTRU SEA CLUTTER
% ============================================================================
\chapter{Adaptări pentru sea clutter}
\label{chap:adaptations}

Modificări cheie pentru adaptarea algoritmului din paper la sea clutter real:

\section{Adaptarea 1: K-distribution în loc de Gaussian}

Algoritmul original presupune statistică Gaussian/Rayleigh, însă sea clutter real are cozi grele și urmează \textbf{K-distribution}:

\begin{equation}
p(x) = \frac{4}{\Gamma(\nu)} \left(\frac{\nu x^2}{2\mu}\right)^{(\nu+1)/2} K_{\nu-1}\left(\sqrt{\frac{2\nu x^2}{\mu}}\right)
\end{equation}

unde $\nu$ e parametrul de formă (mai mic $\rightarrow$ mai „spiky"), iar $K_{\nu-1}$ e funcția Bessel modificată.

\textbf{Soluție}: Estimare parametru din celulele de training: $\nu = \mu^2 / (\sigma^2 - \mu^2)$. Apoi ajustăm multiplicatorul pragului pentru a reduce alarmele false în clutter cu cozi grele.

\section{Adaptarea 2: Fractal boost cu exponentul Hurst}

CFAR poate rata ținte slabe sub prag. Sea clutter are proprietate fractală (self-similarity) cuantificată de exponentul Hurst:

\begin{equation}
\mathbb{E}\left[|X(t+\tau) - X(t)|^2\right] \propto \tau^{2H}
\end{equation}

Pentru sea clutter: $H \approx 0.75$--$0.85$ (persistent). Când apare o țintă, structura se perturbă și $H$ scade sub $\sim$0.6.

\textbf{Soluție}: Calculăm $H$ per bin de frecvență (analiză R/S). Dacă $H < 0.6$, marcăm punctele cu putere mare ca ținte potențiale. Combinăm cu CFAR: $\text{mască} = \text{CFAR} \lor (\text{anomalie Hurst} \land \text{putere mare})$. Aceasta îmbunătățește detecția țintelor slabe pe care CFAR singur le-ar rata.

\section{Adaptarea 3: DBSCAN asimetric pentru semnături verticale}

Țintele apar ca \textbf{linii verticale} (multe frecvențe, puține momente). DBSCAN standard fragmentează acestea în clustere multiple.

\textbf{Soluție}: Distanță asimetrică cu \texttt{freq\_scale=3.0}: $d = \sqrt{\Delta t^2 + (\Delta f / 3)^2}$.

\textbf{Impact}: Un target întins pe $\sim$50 bin-uri de frecvență devine un singur cluster.

\section{Adaptarea 4: Mascare componentă DC}

Componenta DC (0 Hz) produce detecții persistente de la returnări staționare de la valuri.

\textbf{Soluție}: Mascăm $\pm 8$ bin-uri de frecvență în jurul DC înainte de CFAR.

\textbf{Justificare}: DC reprezintă returnări staționare; țintele în mișcare au Doppler non-zero.

\section{Adaptarea 5: Filtru de lățime Doppler}

Unele alarme false apar ca detecții foarte înguste (single-frequency), implauzibil fizic.

\textbf{Soluție}: Respingem clustere cu bandwidth Doppler $< 3$ Hz.

% ============================================================================
% CAPITOLUL 7: DETALII DE IMPLEMENTARE
% ============================================================================
\chapter{Detalii de implementare}

\section{Parametri și calibrare}

\begin{table}[H]
\centering
\caption{Parametrii algoritmului -- valori utilizate}
\begin{tabular}{lccc}
\toprule
\textbf{Parametru} & \textbf{Valoare} & \textbf{Interval} & \textbf{Semnificație} \\
\midrule
$N_{fft}$ (window\_size) & 256 & $[128, 512]$ & Lungime FFT \\
$H$ (hop\_size) & 32 & $[N/8, N/2]$ & Hop = 87.5\% overlap \\
$\sigma_{window}$ & 8 & [4, 16] & Deviația std. a ferestrei \\
$P_f$ & 0.001 & [0.0001, 0.01] & Probabilitate alarmă falsă \\
$N_G$ (cfar\_guard) & 3 & [2, 8] & Mărime celule guard \\
$N_T$ (cfar\_training) & 12 & [8, 24] & Mărime celule training \\
$\varepsilon_{DBSCAN}$ & 8 & [4, 16] & Raza de clustering \\
minSamples & 5 & [3, 10] & Puncte minime per cluster \\
freq\_scale & 3.0 & [2, 5] & Scalare DBSCAN asimetric \\
dc\_mask\_bins & 8 & [4, 16] & Bin-uri DC de mascat \\
min\_doppler\_bw & 3.0 Hz & [1, 10] & Lățime Doppler minimă \\
\bottomrule
\end{tabular}
\end{table}

\section{Interpretare Doppler}

Conversie Doppler $\rightarrow$ viteză radială:
\begin{equation}
v_r = \frac{f_d \cdot c}{2 f_{RF}}
\end{equation}

Pentru IPIX ($f_{RF} = 9.39$ GHz): $f_d = +100$ Hz $\rightarrow$ $v_r \approx +1.6$ m/s; $f_d = -100$ Hz $\rightarrow$ $v_r \approx -1.6$ m/s.

Viteza maximă neambiguă:
\begin{equation}
v_{max} = \frac{PRF \cdot c}{4 f_{RF}} \approx \pm 8 \text{ m/s}
\end{equation}

\section{Dependențe software}

\begin{itemize}
    \item \textbf{NumPy} $\geq$ 1.19 -- operații matriciale
    \item \textbf{SciPy} $\geq$ 1.5 -- STFT, convoluție, iFFT
    \item \textbf{matplotlib} $\geq$ 3.3 -- vizualizări
    \item \textbf{Pillow} -- extragere cadre GIF
    \item (opțional) \textbf{scikit-learn} -- referință DBSCAN
\end{itemize}

% ============================================================================
% CAPITOLUL 8: CONCLUZII
% ============================================================================
\chapter{Concluzii și direcții viitoare}

\section{Concluzii principale}

Implementarea CFAR-STFT demonstrează performanță excelentă:

\begin{enumerate}
    \item \textbf{Acuratețe}: RQF = 29.17 dB @ SNR=30dB, detecție 100\%
    \item \textbf{Robustețe}: Performanță consistentă pe 100 rulări MC
    \item \textbf{Eficiență computațională}: $\sim 75$ ms per segment (13 FPS)
    \item \textbf{Reproductibilitate}: Cod open-source, rezultate verificabile
    \item \textbf{Validare reală}: Funcționează pe date IPIX sea clutter complexe
\end{enumerate}

\section{Perspective de dezvoltare viitoare}

\begin{itemize}
    \item \textbf{Accelerare GPU}: Implementare CUDA/OpenCL pentru timp real
    \item \textbf{Optimizare parametri}: Adaptare automată CFAR în funcție de tipul semnalului
    \item \textbf{Sistem real}: Integrare cu radar operațional
    \item \textbf{Multi-target}: Tracking și predicție traiectorii
    \item \textbf{Machine learning}: Calibrare automată a parametrilor
\end{itemize}

\section{Disponibilitate cod}

\textbf{Repository GitHub}: \url{https://github.com/dirgnic/Radar_Detection_STFT}

Toate fișierele de cod, date și rezultate sunt disponibile public pentru reproductibilitate și replicare independentă.

\begin{thebibliography}{99}

\bibitem{abratkiewicz2022}
Abratkiewicz, K. (2022).
\textit{Radar Detection-Inspired Signal Retrieval from the Short-Time Fourier Transform}.
Sensors, 22(16), 5954.
\newline \url{https://doi.org/10.3390/s22165954}

\bibitem{ipix}
S. Haykin, et al.,
``IPIX Radar Database,''
McMaster University / DREO, 1993.
\url{http://soma.ece.mcmaster.ca/ipix/}

\bibitem{ward2006}
K. D. Ward, R. J. A. Tough, S. Watts,
\textit{Sea Clutter: Scattering, the K Distribution and Radar Performance},
IET, 2006.

\bibitem{hurst1951}
H. E. Hurst,
``Long-term storage capacity of reservoirs,''
\textit{Trans. Am. Soc. Civil Eng.}, vol. 116, pp. 770-799, 1951.

\bibitem{harris1978}
Harris, F.J. (1978).
\textit{On the Use of Windows for Harmonic Analysis with the Discrete Fourier Transform}.
Proceedings of the IEEE, 66(1), 51-83.

\bibitem{ester1996}
Ester, M., Kriegel, H.-P., Sander, J., Xu, X. (1996).
\textit{A Density-Based Algorithm for Discovering Clusters in Large Spatial Databases with Noise}.
In KDD'96: Proceedings, pp. 226-231.

\bibitem{rohling1983}
H. Rohling,
``Radar CFAR thresholding in clutter and multiple target situations,''
\textit{IEEE Trans. Aerospace Electron. Syst.}, vol. 19, no. 4, 1983.

\bibitem{richards2005}
Richards, M. A. (2005).
\textit{Fundamentals of Radar Signal Processing}.
McGraw-Hill Professional.

\end{thebibliography}

\end{document}
