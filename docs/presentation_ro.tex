% Prezentare CFAR-STFT Detecția Radarului
% Autori: Ingrid Corobana, Teodora Nae
% Compilare: pdflatex presentation_ro.tex

\documentclass[12pt,aspectratio=169]{beamer}

\usepackage[utf8]{inputenc}
\usepackage[romanian]{babel}
\usepackage{amsmath}
\usepackage{amssymb}
\usepackage{algorithm}
\usepackage{algpseudocode}
\usepackage{graphicx}
\usepackage{tikz}
\usepackage{listings}
\usepackage{xcolor}
\usepackage{hyperref}

% Definire temă
\usetheme{Madrid}
\usecolortheme{default}

\definecolor{stftblue}{RGB}{66, 133, 244}
\definecolor{cfarorange}{RGB}{255, 152, 0}
\definecolor{dbscangreen}{RGB}{76, 175, 80}
\definecolor{darkblue}{RGB}{25, 50, 100}

\setbeamercolor{structure}{fg=darkblue}
\setbeamercolor{alerted text}{fg=cfarorange}
\setbeamercolor{example text}{fg=dbscangreen}

\lstset{
    language=Python,
    basicstyle=\ttfamily\tiny,
    breaklines=true,
    keywordstyle=\color{blue},
    commentstyle=\color{gray},
    stringstyle=\color{red},
    identifierstyle=\color{darkblue},
    numbers=left,
    numberstyle=\tiny
}

	itle[CFAR-STFT]{Analiza semnalelor radar în prezența ecourilor marine (sea clutter)}
\subtitle{Abordare CFAR-STFT, experimente pe date sintetice și IPIX}
\author[Ingrid Corobana, Teodora Nae]{Ingrid Corobana \and Teodora Nae}
\date[\today]{\today}
\institute[Anul III]{Universitatea din București\\Facultatea de Matematică și Informatică}

\begin{document}

% Slide 1: Titlu
\frame{\titlepage}

% Slide 2: Agenda
\begin{frame}
\frametitle{Agenda}
\tableofcontents
\end{frame}

% Secțiunea 1: Introducere
\section{Introducere și Motivație}

\begin{frame}
\frametitle{Problema Detecției Radar}
\begin{itemize}
    \item \textbf{Provocare}: Detecția fiabilă de semnale radar în sea-clutter
    \item \textbf{Dificulți tradiționale}:
    \begin{enumerate}
        \item Metode CFAR clasice - doar domeniu frecvență
        \item Pierdere informații temporale
        \item Rate înalte de fals alarm în clutter
    \end{enumerate}
    \item \textbf{Soluție}: Exploatare structură timp-frecvență cu CFAR-STFT
    \item \textbf{Referință}: Abratkiewicz (2022) - Sensors
\end{itemize}
\end{frame}

\begin{frame}
\frametitle{Obiectivele Proiectului}
\begin{enumerate}
    \item Implementare completă algoritm CFAR-STFT în Python
    \item Validare pe date sintetice (chirp nonliniar)
    \item Testare pe date reale (radar IPIX X-band)
    \item Analiză Doppler pentru viteza obiectelor
    \item Documentație detaliată și reproducibilitate
\end{enumerate}

\vspace{1em}

\textbf{Rezultate așteptate}:
\begin{itemize}
    \item Detecție 100\% pe semnale sintetice
    \item RQF (Reconstruction Quality Factor) > 25 dB
    \item Timp execuție < 100 ms/segment
\end{itemize}
\end{frame}

% Secțiunea 2: Algoritm
\section{Algoritmul CFAR-STFT}

\begin{frame}
\frametitle{Pipeline Algoritm - Pasul 1: STFT}
\begin{itemize}
    \item \textbf{Transformata Fourier cu Timp Scurt}:
    \[
    X(k,n) = \sum_{m=0}^{N-1} x(m) \cdot w(m - nH) \cdot e^{-j2\pi km/N}
    \]
    \item \textbf{Parametri STFT}:
    \begin{itemize}
        \item FFT length: $N=512$
        \item Hop: $H=256$ (50\% suprapunere)
        \item Fereastră: Gaussiană cu $\sigma=8$ bin-uri
    \end{itemize}
    \item \textbf{Avantaj}: Reprezentare timp-frecvență localizată
\end{itemize}
\end{frame}

\begin{frame}
\frametitle{Pipeline Algoritm - Pasul 2: CFAR 2D}
\begin{itemize}
    \item \textbf{Constant False-Alarm Rate} - adaptat la 2D:
    \[
    H(k,n) = \begin{cases}
    1 & \text{dacă } |X(k,n)|^2 > P_f \cdot \mathcal{N}(k,n) \\
    0 & \text{altfel}
    \end{cases}
    \]
    \item \textbf{GOCA-CFAR (Guard-Cell Order-Statistic)}:
    \[
    \mathcal{N}(k,n) = \frac{1}{4N_G N_T} \sum_{TC} |X(i,j)|^2
    \]
    \item \textbf{Parametri}:
    \begin{itemize}
        \item $P_f = 0.4$ (probabilitate falsă alarmă)
        \item $N_G = N_T = 16$ (guard + training cells)
    \end{itemize}
\end{itemize}
\end{frame}

\begin{frame}
\frametitle{Pipeline Algoritm - Pasul 3: DBSCAN}
\begin{itemize}
    \item \textbf{Density-Based Spatial Clustering}:
    \begin{itemize}
        \item Grupare puncte detectate în planul timp-frecvență
        \item Rază clustering: $\varepsilon = 4$ Hz/s
        \item Minim puncte: minSamples $= 5$
    \end{itemize}
    \item \textbf{Avantaj}: Elimină zgomot și izolate (false alarme)
    \item \textbf{Rezultat}: Clustere coerente = semnale reale
\end{itemize}
\end{frame}

\begin{frame}
\frametitle{Pipeline Algoritm - Pasul 4-5: Dilatare + iSTFT}
\begin{columns}[T]
\column{0.48\textwidth}
\textbf{Pasul 4: Dilatare Geodezică}
\begin{itemize}
    \item Expandare mască cu 3-5 iterații
    \item Recuperare energie pierdută în mascare
    \item Kernel: cruce $3 \times 3$ binar
\end{itemize}

\column{0.48\textwidth}
\textbf{Pasul 5: iSTFT cu Mască}
\begin{itemize}
    \item Reconstrucție inversă: $\hat{x}(n) = \text{iFFT}(X_{masked})$
    \item Overlap-add cu fereastră identică
    \item Normalizare finală
\end{itemize}
\end{columns}

\vspace{1.5em}
\textbf{Metrica Evaluare}: Reconstruction Quality Factor (RQF)
\[
\text{RQF} = 10 \log_{10}\left(\frac{\sum_m |x(m)|^2}{\sum_m |x(m) - \hat{x}(m)|^2}\right) \text{ [dB]}
\]
\end{frame}

% Secțiunea 3: Rezultate
\section{Rezultate Experimentale}

\begin{frame}
\frametitle{Rezultate - Semnale Sintetice}
\begin{table}[h]
\centering
\scalebox{0.9}{
\begin{tabular}{cccc}
\hline
\textbf{SNR [dB]} & \textbf{RQF\_mean [dB]} & \textbf{RQF\_std} & \textbf{P\_detect [\%]} \\
\hline
5 & 7.28 & 0.47 & 100 \\
10 & 16.81 & 0.60 & 100 \\
15 & 22.95 & 0.56 & 100 \\
20 & 26.40 & 0.51 & 100 \\
25 & 28.43 & 0.39 & 100 \\
30 & 29.17 & 0.25 & 100 \\
\hline
\end{tabular}
}
\end{table}

\textbf{Observații}:
\begin{itemize}
    \item 100 rulări Monte Carlo per SNR level
    \item Detecție perfectă (100\%) la toate nivelurile
    \item RQF crește monoton cu SNR
    \item Variabilitate scade la SNR înalt
\end{itemize}
\end{frame}

\begin{frame}
\frametitle{Rezultate - Date Reale IPIX}
\textbf{Test pe 50 segmente X-band Radar (9.39 GHz)}

\begin{itemize}
    \item Sursă: IPIX database (Canadian institute research)
    \item Tip semnal: Complex I/Q sea-clutter
    \item Rezultate:
    \begin{itemize}
        \item \textbf{Detecții}: 3.2 obiecte/segment (mediu)
        \item \textbf{Viteza Doppler}: 2.4 m/s ± 0.8 m/s
        \item \textbf{False alarme}: < 1\%
        \item \textbf{Timp execuție}: 75 ms/segment
    \end{itemize}
\end{itemize}

\vspace{1em}
\textbf{Concluzie}: Algoritmul generalizează bine de la date sintetice la reale.
\end{frame}

\begin{frame}
\frametitle{Performanță Computațională}
\begin{itemize}
    \item \textbf{STFT} (FFT optimizat): 10 ms
    \item \textbf{CFAR 2D} (cu convoluție): 50 ms
    \item \textbf{DBSCAN}: 5 ms
    \item \textbf{iSTFT}: 8 ms
    \item \textbf{Dilatare}: 2 ms
    \vspace{0.5em}
    \item \textbf{Total}: \textbf{75 ms/segment} $\Rightarrow$ \textbf{13 FPS}
\end{itemize}

\vspace{1em}
\textbf{Interpretare}: Suficient pentru aplicații offline și semi-real-time.
\end{frame}

% Secțiunea 4: Implementare
\section{Detalii de Implementare}

\begin{frame}
\frametitle{Structura Cod}
\textbf{src/cfar\_stft\_detector.py (1110 linii)}:
\begin{itemize}
    \item Clasa \texttt{CFAR2D}: Detecție adaptivă 2D
    \item Clasa \texttt{DBSCAN}: Clustering custom
    \item Clasa \texttt{CFARSTFTDetector}: Pipeline complet
    \item Clasa \texttt{AcousticCFARDetector}: Variantă acustică
\end{itemize}

\textbf{simulations/paper\_replication.py (1087 linii)}:
\begin{itemize}
    \item \texttt{run\_paper\_experiment()}: Reproducere articol
    \item \texttt{run\_ipix\_experiment()}: Date reale
    \item \texttt{compute\_rqf()}: Metrica evaluare
\end{itemize}

\textbf{Dependințe}:
\begin{itemize}
    \item NumPy, SciPy, matplotlib
    \item Fără framework-uri heavy (TensorFlow, PyTorch)
\end{itemize}
\end{frame}

\begin{frame}[fragile]
\frametitle{Exemplu Cod - STFT}
\begin{lstlisting}
def compute_stft(x, N_fft=512, hop=256, sigma=8):
    N_t = (len(x) - N_fft) // hop + 1
    X_stft = np.zeros((N_fft//2 + 1, N_t), dtype=complex)
    
    # Fereastră Gaussiană
    window = np.exp(-np.arange(N_fft)**2 / (2*sigma**2))
    window /= np.sqrt(np.sum(window**2))
    
    for n in range(N_t):
        x_n = x[n*hop : n*hop + N_fft] * window
        X_n = np.fft.rfft(x_n, N_fft)
        X_stft[:, n] = X_n
    
    return X_stft
\end{lstlisting}
\end{frame}

\begin{frame}[fragile]
\frametitle{Exemplu Cod - CFAR 2D}
\begin{lstlisting}
def detect_cfar2d(X_stft, P_f=0.4, N_G=16, N_T=16):
    H = np.zeros(X_stft.shape, dtype=bool)
    N_f, N_t = X_stft.shape
    
    for k in range(N_G + N_T, N_f - N_G - N_T):
        for n in range(N_G + N_T, N_t - N_G - N_T):
            # Training cells
            X_tc = X_stft[k-N_G-N_T:k-N_G, ...]
            N_local = np.mean(np.abs(X_tc)**2)
            
            lambda_th = P_f * N_local
            if np.abs(X_stft[k,n])**2 > lambda_th:
                H[k, n] = True
    
    return H
\end{lstlisting}
\end{frame}

% Secțiunea 5: Concluzii
\section{Concluzii}

\begin{frame}
\frametitle{Contribuții Principale}
\begin{enumerate}
    \item \textbf{Implementare completă}: CFAR-STFT end-to-end în Python
    \item \textbf{Validare}:
    \begin{itemize}
        \item Semnale sintetice: 100\% detecție, RQF = 29.17 dB @ SNR=30dB
        \item Semnale reale: Funcționare pe date IPIX complexe
    \end{itemize}
    \item \textbf{Reproducibilitate}: Cod open-source, rezultate verificabile
    \item \textbf{Performanță}: 75 ms/segment (13 FPS)
\end{enumerate}
\end{frame}

\begin{frame}
\frametitle{Perspective Viitoare}
\begin{itemize}
    \item Accelerare GPU (CUDA/OpenCL)
    \item Optimizare parametri adaptivi
    \item Integrare sisteme radar real-time
    \item Multi-target tracking și predictie traiectorii
    \item Machine learning pentru calibrare automată
\end{itemize}

\vspace{1.5em}

\textbf{Repository}: \url{https://github.com/dirgnic/Radar_Detection_STFT}

\textbf{Status}: \alert{COMPLET} - Gata pentru implementare și desfășurare
\end{frame}

\begin{frame}
\frametitle{Mulțumiri}
\begin{itemize}
    \item Recunoștiință: Abratkiewicz (2022) - paper original
    \item Mulțumiri: Baza IPIX - date experimentale
    \item Cod disponibil: GitHub public repository
\end{itemize}

\vspace{2em}

\begin{center}
{\Large \textbf{Întrebări?}}
\end{center}
\end{frame}

\end{document}
