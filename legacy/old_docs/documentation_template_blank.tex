% Documentație conform șablonului FMI (report, Times, 1.5 line spacing)
% Șablon gol de pornire pentru completare
% Compilare recomandată: xelatex documentation_template_blank.tex

\documentclass[12pt,a4paper]{report}

% Suport diacritice și font Times (șablon FMI)
\usepackage{fontspec}
\usepackage{times}
\usepackage{polyglossia}
\setdefaultlanguage{romanian}
\setotherlanguages{english}
\SetLanguageKeys{romanian}{indentfirst=true}

\usepackage{geometry}
\usepackage{setspace}
\doublespacing
\geometry{margin=2.5cm}
\usepackage{fancyhdr}
\usepackage{graphicx}
\usepackage{amsmath}
\usepackage{amssymb}
\usepackage{algorithm}
\usepackage{algpseudocode}
\usepackage{listings}
\usepackage{xcolor}
\usepackage{hyperref}
\usepackage{tikz}
\usepackage{array}
\usepackage{booktabs}
\usepackage{float}
\usepackage{caption}
\usepackage{pdfpages} % opțional: includeți pipeline_diagrams.pdf

\geometry{
    a4paper,
    left=2.5cm,
    right=2.5cm,
    top=2.5cm,
    bottom=2.5cm,
    headheight=14pt
}

\pagestyle{fancy}
\fancyhf{}
\fancyhead[L]{\textit{Titlu scurt lucrare}}
\fancyhead[R]{\thepage}
\renewcommand{\headrulewidth}{0.4pt}

\definecolor{stftblue}{RGB}{66, 133, 244}
\definecolor{cfarorange}{RGB}{255, 152, 0}
\definecolor{dbscangreen}{RGB}{76, 175, 80}
\definecolor{darkblue}{RGB}{25, 50, 100}
\definecolor{codegray}{RGB}{240, 240, 240}

\hypersetup{
    colorlinks=true,
    linkcolor=darkblue,
    urlcolor=stftblue,
    bookmarksnumbered=true,
    pdftitle={Titlu lucrare}
}

\lstset{
    language=Python,
    basicstyle=\ttfamily\small,
    backgroundcolor=\color{codegray},
    breaklines=true,
    keywordstyle=\color{blue},
    commentstyle=\color{gray},
    stringstyle=\color{red},
    identifierstyle=\color{darkblue},
    numbers=left,
    numberstyle=\tiny\color{gray}
}

\renewcommand{\algorithmicrequire}{\textbf{Intrare:}}
\renewcommand{\algorithmicensure}{\textbf{Ieșire:}}

\title{
    \textbf{\Large Titlu lucrare (în limba română)}\\[0.5em]
    \normalsize Subtitlu opțional (problemă/abordare/experiment)
}

\author{
    Nume Prenume\\
    \small \textit{Anul III, Grupa XX}\\[0.3em]
    \href{https://github.com/username/repo}{\texttt{github.com/username/repo}}
}

\date{\today}

\begin{document}

\maketitle

\begin{abstract}
Rezumat scurt al lucrării: context, obiectiv, metodă, rezultate principale și contribuții. 150--250 de cuvinte.

\textbf{Cuvinte-cheie}: termen1, termen2, termen3
\end{abstract}

\newpage
\tableofcontents
\newpage

% Secțiune opțională: diagrame pipeline
\section*{Diagrame Pipeline}
\addcontentsline{toc}{section}{Diagrame Pipeline}
Includeți aici descrierea diagramelor. Dacă aveți fișier PDF generat separat, decomentați figura de mai jos.

% \begin{figure}[H]
%     \centering
%     \includepdf[pages=-,scale=0.9,pagecommand={}]{pipeline_diagrams.pdf}
%     \caption*{Diagramele pipeline (preluate din pipeline_diagrams.pdf).}
% \end{figure}

\newpage

\section{Introducere}
Context, motivație, problemă abordată, contribuții.

\subsection{Obiective}
Lista obiectivelor numerotate.

\subsection{Structura lucrării}
Scurtă descriere a secțiunilor.

\section{Fundamente Teoretice}
Concepte, definiții, ecuații relevante. Exemplu de ecuație:
\begin{equation}
X(k,n) = \sum x(m) w(m-nH) e^{-j2\pi km/N}
\end{equation}

\section{Metodologie / Algoritm}
Descriere pas cu pas; puteți folosi pseudocod:
\begin{algorithm}[H]
\caption{Numele algoritmului}
\begin{algorithmic}[1]
    \Require Intrări
    \Ensure Ieșiri
    \State // pași ai algoritmului
\end{algorithmic}
\end{algorithm}

\section{Rezultate Experimentale}
Date, grafice, tabele (folosiți booktabs pentru tabele).

\section{Implementare și Detalii Tehnice}
Referințe către fișierele sursă, decizii de design, parametri.

\section{Concluzii și Lucrări Viitoare}
Recapitulare contribuții și pași de îmbunătățire.

\begin{thebibliography}{9}
\bibitem{exemplu} Autor, Titlu, Revistă/Conferință, An.
\end{thebibliography}

\end{document}
